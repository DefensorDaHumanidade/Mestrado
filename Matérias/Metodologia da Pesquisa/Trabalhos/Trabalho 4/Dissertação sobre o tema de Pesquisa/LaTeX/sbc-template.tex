\documentclass[12pt]{article}
\usepackage{sbc-template}
\usepackage[alf]{abntex2cite} 
\usepackage{graphicx,url}
\usepackage[utf8]{inputenc}
\usepackage[brazil]{babel}
\usepackage{listings}
\usepackage{minted}
%\usepackage[latin1]{inputenc}      %Erro no inputenc.

\sloppy

\title{RESENHA: \\ Um Jogo como Estratégia de Apoio à \\ Prevenção da Violência Sexual Infantil}

\author{Alexandre Mendonça Fava\inst{1}}

\address{\inst{}Universidade do Estado de Santa Catarina - UDESC\\Programa de Pós-graduação em Computação Aplicada - PPGCA\\Joinville - SC - Brasil CEP: 89.219-710
    \email{alexandre.fava@hotmail.com}
}

\begin{document} 

\maketitle

A questão da violência sexual é um problema que assola a humanidade há séculos. As maiores vítimas da violência sexual da atualidade são as crianças. Em resposta, inúmeras soluções e estratégias de mitigação a este mal foram desenvolvidas. Dentre as iniciativas de combate e prevenção a violência sexual infantil, estão os jogos sérios. 

Tiago Francisco Andrade Diocesano, é o pioneiro (de acordo com a literatura pesquisada) no que diz respeito ao desenvolvimento de um jogo sério como estratégia de apoio à prevenção da violência sexual infantil, a nível nacional. Cabe salientar que iniciativas educacionais do gênero já encontram-se devidamente documentadas e já são aplicadas em outros países ao redor do globo. Em virtude da inexistência de um jogo nacional e em virtude da magnitude do problema da violência sexual infantil no Brasil, Tiago Diocesano trouxe está solução para as terras tupiniquins.

A dissertação do Tiago foi publicada no final de 2018. Nela, são listados vários jogos objetivados no combate da violência sexual infantil. Nenhum dos jogos apresentados pelo Tiago possui tradução em português, o que acabou por se tornar mais uma justificativa para o desenvolvimento de um jogo totalmente brasileiro, o qual recebeu o nome de \textbf{Infância Segura}. Na fundamentação de sua dissertação, Tiago detalha alguns conceitos educacionais que são ministrados no jogo, porém não fica claro como estes conceitos foram elencados. Em adendo, o guia de `Orientações técnicas internacionais de educação em sexualidade' elencada alguns conceitos apresentados pelo Tiago, o que demonstra uma escolha dos conceitos bem fundamentada, porém não muito justificada.

Na seção dos trabalhos relacionados da dissertação do Tiago, são apresentados várias aplicações voltados para o combate e prevenção da violência sexual infantil, com algumas aplicações sendo jogos e outras não. Acessando as aplicações é possível constatar uma certa incongruência entre suas respectivas documentações e a dissertação do Tiago Diocesano, isso pois algumas aplicações não classificadas como jogos pelo Tiago, são relatadas como sendo jogos em suas documentações oficiais. Essa incoerência de informações possui duas justificativas: ou a aplicação foi atualizada (após a publicação da dissertação, mudando sua característica para jogo), ou houve um equívoco do Tiago no momento de redigir essa informação.

Na seção do desenvolvimento, Tiago relata o processo de criação dos protótipos do jogo. Seus questionários e entrevistas com especialistas da área de educação sexual são bem documentos, contudo nota-se uma carência documental acerca do processo de desenvolvimento do jogo em si. Salienta-se que Tiago apresenta a arquitetura do jogo, relata as ferramentas usadas, as artes utilizadas, contudo não há menção de uma metologia de desenvolvimento (engenharia de \textit{software}). Durante a dissertação, Tiago até que menciona um pouco sobre o Design de Interação, o Design Participativo e alguns modelos para classificação de jogos, contudo o processo de desenvolvimento do jogo em si não é abordado. Implicitamente durante o texto, tem-se a impressão que a metodologia PEED está sendo executada, contudo como não há citação dela na fundamentação da dissertação e não há a citação de um GDD, por tal razão, nada se pode afirmar sobre a metologia de desenvolvimento do \textbf{Infância Segura}. 

Essa resenha crítica poderia se estender por mais de 100 páginas, afinal a avaliação da dissertação do Tiago e seu jogo já foram avaliados no Trabalho de Conclusão de Curso (TCC) do presente autor. Na monografia (TCC), além das críticas aqui já elencadas, uma avaliação sobre o jogo é realizada, revelando uma série de problemas de usabilidade e ergonomia no jogo (além de problemas na abordagem educativa). A pesquisa sobre o \textbf{Infância Segura} continua, afinal objetiva-se trazer a melhor experiência possível para as crianças, evitando desta forma quaisquer traumas que um jogo mal projetado poderia causar. 


\begin{thebibliography}{9}

\bibitem{DIOCESANO} 
Diocesano, Tiago Francisco Andrade.
\textit{Um jogo como Estratégia de apoio à Prevenção da Violência Sexual Infantil}.
Universidade do Estado de Santa Catarina, Dissertação de Mestrado (2018).

\bibitem{ALEXANDRE} 
Fava, Alexandre Mendonça.
\textit{Avaliação e Adaptação do Jogo Sério Infância Segura}.
Universidade do Estado de Santa Catarina, Trabalho de Conclusão de Curso (2019).

\end{thebibliography}

\end{document}