\documentclass[12pt]{article}
\usepackage{sbc-template}
\usepackage[alf]{abntex2cite} 
\usepackage{graphicx,url}
\usepackage[utf8]{inputenc}
\usepackage[brazil]{babel}
\usepackage{framed}                 %Incluindo Caixas
%\usepackage[latin1]{inputenc}      %Erro no inputenc.

\sloppy

\title{Metodologia da Pesquisa: \\Um Jogo Sério para prevenção da Violência Sexual Infantil}

\author{Alexandre Mendonça Fava\inst{1}}

\address{\inst{}Universidade do Estado de Santa Catarina - UDESC\\Programa de Pós-graduação em Computação Aplicada - PPGCA\\Joinville - SC - Brasil CEP: 89.219-710
    \email{\{alexandre.fava@hotmail.com\}}
}

\begin{document} 

\maketitle

\vspace{1cm}

\section{Metodologia}\label{secao:introducao}

O presente trabalho assume uma proposta de \textbf{pesquisa positivista} de \textbf{natureza aplicada}, pois sua fundamentação tem base no desenvolvimento de um jogo sério que tem aplicação prática e direta na prevenção da violência sexual infantil \cite{lacerda2009augusto}. A validação do \textit{software} (jogo) ocorrerá por meio da extração de aspectos humanos difíceis de mensurar ou medir. Taxas de acerto e sucesso dos jogadores durante o jogar do jogo serão variáveis empíricas bem fundamentas a serem colhidas. Contudo, as questões de retenção do conhecimento e engajamento, que são o cerne da atual pesquisa, ainda carecem de métodos exatos de medição. 

Além disso, é de \textbf{abordagem de pesquisa quanti-qualitativa}, pois essa pesquisa busca basear-se tanto em dados quantitativos quanto em dados qualitativos \cite{ludke2011pesquisa}. Em questionários pensados para a avaliação por exemplo (questões fechadas), as respostas podem ser observadas tanto estatisticamente quanto subjetivamente. Em adendo, critérios de desempenho são variáveis quantitativas a serem colhidas no decorrer da pesquisa \cite{souza2017abordagem}. Sendo assim, a \textbf{forma das variáveis} assume tanto caráter cardinal, quanto categórico. Ou seja, os dados de desempenho no jogo são informações quantitativas cardinais por assumirem valores bem definidos e estabelecidos, enquanto os dados colhidos nas entrevistas e questionários são informações qualitativas que refletem uma pouco da opinião sobre o jogo.

O principal objetivo da pesquisa é constatar a \textbf{dependência} ou \textbf{independência} das variáveis de desempenho e retenção do conhecimento. Em outras palavras, busca-se identificar se os conceitos abordados no jogo são absorvidos pelos jogadores. Tal objetivo pode ser resumido na seguinte \textbf{hipótese} de pesquisa: Jogos Sérios projetados na temática preventiva à violência sexual infantil fortalecem o aprendizado infantil acerca das formas de prevenção ao abuso sexual infantil. A longo prazo, caso a hipótese seja confirmada, espera-se uma redução no índices de violência sexual infantil a nível nacional nos anos seguintes a implementação do jogo. Contudo, essa é uma pesquisa de \textbf{recorte transversal}, com tempo máximo de dois anos de investigação, não se podendo averiguar melhoras no quadro criminal brasileiro no que diz respeito aos crimes de cunho sexual (em um prazo curto de dois anos). 

\newpage

Existem soluções baseadas em jogos para o combate a violência sexual infantil. A atual pesquisa não se busca a mensurar as diferenças entre as soluções presentes na literatura, o objetivo seria o de apresentar algo diferente as soluções já existentes, sendo este portanto o \textbf{nível de maturidade} do presente trabalho. 

Quanto ao \textbf{método de pesquisa}, o atual trabalho configura-se como \textbf{estudo de caso}, pois realizou-se levantamento bibliográfico, entrevistas e empregou-se técnicas de análise estatística \cite{freitas2000metodo}. A \textbf{finalidade} desta pesquisa se caracteriza como uma pesquisa descritiva. A pesquisa descritiva pode ser realizada para descrever como um grupo realiza um determinado trabalho, como usa um determinado sistema para a realização de tarefas, ou como o sistema é usado por diferentes perfis pessoais \cite{fontelles2009metodologia}. Em linhas gerais, um estudo de caso é positivista quando o mesmo é estruturado com determinação prévia e objetiva de variáveis de investigação e interpretativista quando é estruturado com técnicas de coleta mais qualitativas \cite{saccol2009retorno}.

O estudo de caso é indicado quando não é possível separar o fenômeno do ambiente real, quando não são conhecidas todas as variáveis relevantes relacionadas com o fenômeno. Para \citeonline{berkenbrockprevenccao}, no estudo de caso o pesquisador usa vários dados extraídos de diversas fontes de informação em diferentes momentos. A técnica fundamental de pesquisa é a observação e a entrevista que produzem relatórios normalmente informais, narrativos, ilustrados com citações, exemplos e descrições fornecidos pelos sujeitos, podendo também utilizar fotos, desenhos etc.

A \textbf{técnica de coleta de dados} será feita mediante questionários e observações. Para medir as respostas dos questionários (questões fechadas), pretende-se utilizar a Escala Likert. Nas observações, busca-se conhecer os jogadores, assim como as atividades realizadas por eles \cite{dias2000grupo}. Para a \textbf{técnica de análise de dados}, pretende-se utilizar a análise estatística e análise de conteúdo, gerando desta forma gráficos para a compreensão mais intuitiva das informações. Deste modo, a documentação será realiza de maneira \textbf{direta}, a qual, não há a inclusão (ou comparação) de dados de terceiros. 

Em virtude da natureza das informações colhidas, essa pesquisa se configura como uma \textbf{ciência inexata}. As ciência inexatas não são muito comuns na área computacional, porém se alastram pela maioria das ciência sociais \cite{wazlawick2010reflexao}. O foco da atual pesquisa não se encontra na parte computacional do desenvolvimento do \textit{software} (jogo), mas sim na interação que os jogadores tem com o jogo e a influência disso sob seu comportamento. Isso faz com que essa pesquisa seja classificada como uma pesquisa de \textbf{ciência mole} (\textit{soft}).

Por fim, destaca-se que na computação são poucas as subáreas  de \textbf{ciência idiográficas}.  Em especial,  o desenvolvimento  de  determinadas  tecnologias,  delimita uma dessas subáreas \cite{wazlawick2010reflexao}. Desta forma, dada a criação de um artefato computacional, em específico, jogo sério, essa pesquisa se classifica entre as ciência idiográficas.

%A \textbf{posição epistemológica} do presente trabalho é interpretativista pois considera-se que as variáveis da presente pesquisa não se encontram totalmente isoladas %, implicando desta forma que replicações do atual estudo possam apresentar resultados distintos em condições classificadas como similares \cite{fuks2011sistemas}.

\section*{Agradecimentos}\label{secao:agradecimentos}

O presente trabalho foi realizado com apoio da Coordenação de Aperfeiçoamento de Pessoal de Nível Superior - Brasil (CAPES) - Código de Financiamento 001.

%\bibliographystyle{sbc}
\bibliography{sbc-template}

\end{document}