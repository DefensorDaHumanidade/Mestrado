\documentclass[12pt]{article}
\usepackage{sbc-template}
\usepackage[alf]{abntex2cite} 
\usepackage{graphicx,url}
\usepackage[utf8]{inputenc}
\usepackage[brazil]{babel}
\usepackage{listings}
\usepackage{minted}
%\usepackage[latin1]{inputenc}      %Erro no inputenc.

\sloppy

\title{RESENHA: \\ As Estratégias de Combate ao Abuso Sexual Infantil}

\author{Alexandre Mendonça Fava\inst{1}}


\address{\inst{}Universidade do Estado de Santa Catarina - UDESC\\Programa de Pós-graduação em Computação Aplicada - PPGCA\\Joinville - SC - Brasil CEP: 89.219-710
    \email{alexandre.fava@hotmail.com}
}

\begin{document} 

\maketitle


Simon David Finkelhor é um sociólogo estadunidense conhecido por suas pesquisas na temática do abuso sexual infantil. Desde a década de 70 David estuda a questão dos maus-tratos infantis, o que levou a se tornar o atual diretor do Centro de Pesquisa de Crimes contra a Criança. David possui mais de 100 publicações na área, sendo o objeto de análise desta resenha uma publicação de 2009, intitulada: The Prevention of Childhood Sexual Abuse.

Na publicação de 2009, David realiza um rápido panorama acerca das questões da violência contra a criança, dando um enfoque maior as iniciativas para a prevenção do abuso sexual infantil. As iniciativas são o principal objeto de análise desta resenha, uma vez relacionadas diretamente com as pesquisas do presente autor. O presente autor desenvolve um jogo sério para prevenção da violência sexual infantil. Compreender as demais estratégias de combate a este problema é um fator fundamental para compreender as maneiras mais adequadas de enfrentamento e mitigação aos maus tratos infantis. 

A publicação de David é categorizada como um artigo secundário por ser apenas um levantamento das iniciativas existentes. Cada seção apresenta uma estratégia diferente. Após a apresentação da estratégia, seus resultados são apresentados. Resultados estes que são uma mera compilação de achados de outras pesquisas. Deste modo, a pesquisa de David não gera conhecimento novo (artigo primário), apenas compila os resultados e achados de demais pesquisas na área de combate a violência sexual infantil. David não detalha no artigo a forma como realizou sua pesquisa, e nem quais mecanismos de busca utilizou. O presente autor supõe nesse sentido que a omissão destas informações se deve ao largo conhecimento de David na área em questão, conhecimento este, que beira os 50 anos de estudos na área. Conclui-se nesse sentido que todos estes anos de estudos dificilmente seriam resumidos em um artigo de 26 páginas. 

David em sua publicação menciona as estratégias jurídicas, citando algumas leis na temática em questão. Há também a menção de prisões mais rígidas aos agressores sexuais ou ao seu tratamento clínico. Porém, são as políticas educacionais que se demonstram como estratégia mais promissora no combate e enfrentamento ao abuso sexual infantil. David lembra que não há pesquisas suficientes que corroborem a efetividade de uma única iniciativa que lide completamente com a questão da violência infantil, contudo existem resultados extremamente promissores para as iniciativas educacionais. 

\newpage

O artigo explica que os programas escolares parecem ser um sistema eficiente para lidar com as várias formas de abuso. As crianças além de aprenderem a se proteger, também são instruídas a reconhecer e relatar os episódios de violência. David argumenta ainda que tais programas servem de instrumento de dissuasão contra possíveis agressores sexuais.

As escolas são apresentadas pelo artigo como excelentes locais para a execução dos programas de prevenção. Além de sua disponibilidade universal e equitativa as escolas veem assegurar o ensino das crianças na temática em questão de forma mais abrangente. Diferentes locais de ensino poderiam não atingir o mesmo alcance das escolas.

A publicação de David permite a compreensão de uma maneira rasa, porém abrangente na temática em questão. O artigo não atinge o estado da arte requerido pelo presente autor, mas viabiliza um maior esclarecimento sobre as questões dos maus-tratos infantis, demonstrando ainda que o jogo em atual desenvolvimento do autor, deve preferencialmente ser ministrado em salas de aula. 

Outra revisão da literatura se faz necessária para compreender o verdadeiro impacto dos jogos no ensino de conteúdos relacionados a prevenção da violência sexual infantil. Estudar os jogos já existentes na temática em questão se faz fundamental para entender os acertos e os erros já cometidos por pesquisas na área. 

O artigo de David não responde todas as questões de pesquisa para a dissertação do presente autor. Todavia, o artigo de David demonstra que o presente autor está no caminho certo, uma vez constatada a efetividade de estratégias educacionais em relação as demais iniciativas de combate ao abuso sexual infantil. 


\begin{thebibliography}{9}

\bibitem{David} 
Finkelhor, David.
\textit{The prevention of childhood sexual abuse}.
The future of children (2009): 169-194.

\end{thebibliography}

\end{document}