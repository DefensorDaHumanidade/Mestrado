% ---
% RESUMOS
% ---

% resumo em português
\setlength{\absparsep}{8pt} % ajusta o espaçamento dos parágrafos do resumo [aqui estava 18 antes]
\begin{resumo}
   %Elemento obrigatório que contém a apresentação concisa dos pontos relevantes do trabalho, fornecendo uma visão rápida e clara do conteúdo e das conclusões do mesmo. A apresentação e a redação do resumo devem seguir os requisitos estipulados pela NBR 6028 (ABNT, 2003). Deve descrever de forma clara e sintética a natureza do trabalho, o objetivo, o método, os resultados e as conclusões, visando fornecer elementos para o leitor decidir sobre a consulta do trabalho no todo.
  Contexto: A violência sexual infantil é um grave problema de saúde pública. Tal problema traz prejuízos tanto às vítimas, quanto à sociedade. Em resposta a esse fenômeno, inúmeras medidas surgiram ao redor do globo, voltadas à proteção dos direitos das crianças e dos adolescentes. Dentre as medidas, estratégias educativas baseadas em jogos se destacam, demonstrando-se como boas aliadas no ensino preventivo e no combate à violência sexual infantil.

  Objetivo: A presente pesquisa acadêmica realizou o desenvolvimento de um jogo sério focado na ensino-aprendizagem da prevenção da violência sexual. Os conteúdos pedagógicos abordados no jogo são projetados para crianças entre 5 (cinco) e 8 (oito) anos de idade. O principal objetivo do jogo consiste na ideia de educar e conscientizar seus jogadores a identificar e relatar eventos praticados (ou tentados) de violência sexual infantil. Buscando atender o rigor do processo científico, a presente pesquisa realiza um ensaio, de modo a verificar se o jogo em si é capaz de cumprir com seus preceitos pré-estabelecidos.

  Amostra e Cenário: O processo avaliativo do jogo desenvolvido foi realizado com uma amostra de 33 (trinta e três) crianças da Escola Municipal Pauline Parucker. Todos os experimentos envolvendo a amostra de crianças foram conduzidos na própria escola.% (idade média de 8 anos). 

  Método: A amostra de crianças foi segmentada em dois grupos; grupo controle (n = 18) e grupo experimental (n = 15). O grupo experimental foi exposto ao jogo na etapa de teste da presente pesquisa. Ambos os grupos tiveram seus conhecimentos sobre prevenção ao abuso mensurados pelo instrumento avaliativo \ac{CKAQ}. Os conhecimentos dos grupos foram mensurados na etapa de pré-teste e na etapa de pós-teste da corrente pesquisa, de modo a identificar seus níveis de instrução sobre prevenção ao abuso. 

  Resultados: A análise dos dados não apresentou discrepância sobre os níveis de instrução entre os grupos estudados. Não foi possível identificar estatisticamente nível de influência do jogo no que diz respeito ao ensino de seus preceitos pré-estabelecidos (p = 0,27697). Em síntese, ambos os grupos estudados apresentaram graus de conhecimento semelhantes na etapa de pré e pós-teste da corrente pesquisa, no que tange aos conhecimentos medidos pelo instrumento avaliativo \ac{CKAQ}.

  Conclusão: O jogo sério desenvolvido aponta para uma equivalência estatística entre os grupos estudados. Os conceitos pedagógicos do jogo desenvolvido são semelhantes aos conceitos pedagógicos de outros de mesma temática, os quais manifestam eficácia estatística. Um estudo mais aprofundado deve ser conduzido de modo a identificar as causas que culminaram no resultado alcançado. A presente pesquisa acredita que com os devidos ajustes, o jogo possui potencial para assumir um papel chave na prevenção da violência sexual infantil no Brasil.
 

   %A violência sexual infantil é um problema de saúde pública que sequela tanto a vítima quanto a sociedade. Nesse sentido, jogos para prevenção da violência sexual infantil surgem com o objetivo de proporcionarem uma abordagem educacional lúdica acerca do assunto para as crianças. A presente pesquisa introduz um jogo educacional focado na ensino-aprendizagem da prevenção da violência sexual. O jogo desenvolvido segue preceitos pré-estabelecidos internacionalmente para a educação de menores acerca da temática sexual. Dentre os ensinamentos estão conceitos associados ao corpo humano e privacidade corporal, definição dos direitos e deveres, estratégias de segurança pessoal e denúncia, e cuidados a serem tomados na \textit{internet}. O jogo é voltado para crianças de cinco a oito anos de idade. Sua validação se dá por meio do Questionário Sobre Conhecimentos de Abuso Infantil (CKAQ). O questionário é submetido a um grupo experimental e a um grupo controle com o intuito de averiguar diferenças significativas entre os grupos (grau de confiança de 95\%). Tanto o Teste \textit{t}, quanto o teste de \textit{Mann-Whitney} não apontaram para diferenças significas entre os grupos, indicando que não existe significância estatística para afirmar que as crianças tiveram aumento no seus conhecimentos relativos a temátia abordada. 
   
 \textbf{Palavras-chave}: Violência Sexual Infantil; Tecnologia Educacional; Prevenção.
\end{resumo}
