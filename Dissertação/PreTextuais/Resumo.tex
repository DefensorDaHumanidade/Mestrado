% ---
% RESUMOS
% ---

% resumo em português
\setlength{\absparsep}{18pt} % ajusta o espaçamento dos parágrafos do resumo
\begin{resumo}
   %Elemento obrigatório que contém a apresentação concisa dos pontos relevantes do trabalho, fornecendo uma visão rápida e clara do conteúdo e das conclusões do mesmo. A apresentação e a redação do resumo devem seguir os requisitos estipulados pela NBR 6028 (ABNT, 2003). Deve descrever de forma clara e sintética a natureza do trabalho, o objetivo, o método, os resultados e as conclusões, visando fornecer elementos para o leitor decidir sobre a consulta do trabalho no todo.
   A violência sexual infantil é um problema de saúde pública que sequela tanto a vítima quanto a sociedade. Nesse sentido, jogos para prevenção da violência sexual infantil surgem com o objetivo de proporcionarem uma abordagem educacional lúdica acerca do assunto para as crianças. A presente pesquisa introduz um jogo educacional focado na ensino-aprendizagem da prevenção da violência sexual. O jogo desenvolvido segue preceitos pré-estabelecidos internacionalmente para a educação de menores acerca da temática sexual. Dentre os ensinamentos estão conceitos associados ao corpo humano e privacidade corporal, definição dos direitos e deveres, estratégias de segurança pessoal e denúncia, e cuidados a serem tomados na \textit{internet}. O jogo é voltado para crianças de cinco a oito anos de idade. Sua validação se dá por meio do Questionário Sobre Conhecimentos de Abuso Infantil (CKAQ). O questionário é submetido a um grupo experimental e a um grupo controle com o intuito de averiguar diferenças significativas entre os grupos (grau de confiança de 95\%). Caso os resultados apontem para uma diferença entre os grupos, revelando incrementos nas habilidades preventivas do grupo experimental, pode-se dizer que o jogo possui potencial para assumir um papel chave na prevenção da violência sexual infantil no Brasil.

 \textbf{Palavras-chave}: Tecnologia Educacional. Prevenção. Violência Sexual Infantil.
\end{resumo}
