% ---
% Abstract
% ---

% resumo em inglês
\begin{resumo}[Abstract]
 \begin{otherlanguage*}{english}
    Child sexual abuse is a big public health problem. This problem harms both the victims and society. In response, many strategies have emerged around the globe, objectified at protecting the rights of children and adolescents. Among thouse,  educational strategies based on games stand out, proving themselves as good allies in preventive education and in combating child sexual abuse.
    This academic study maked a serious game focused on the prevention of sexual violence. The pedagogical content knowledge was designed for between 5 (five) and 8 (eight) years old. The main objective of the game is the ideia of educating and making its players aware to identify and to report events practiced (or attempted) of child sexual violence. Seeking to meet the rigor of the scientific process, this research reports a academic trial of serious-game for child sexual abuse prevention, in order to verify if the game itself is capable of complying with its pre-established precepts.
    The evaluation involved 33 (thirty-three) children from Escola Municipal Pauline Parucker. All experiments with the children were conducted in the school itself.% (idade media de 8 anos).
    The sample of children was divided into two groups; control group (n = 18) and experimental group (n = 15). The experimental group was exposed to the game during the test phase of this research. All children were pre-tested and post-tested for knowledge of abuse prevention using the \acf{CKAQ}, in order to identify their education levels on abuse prevention.
    Data analysis results didn't show discrepancy about the levels of education among the studied groups.  It was not possible to identify influence level of the game (p = 0.27697). I short, both groups studied showed similar levels of knowledge in the pre-tested and post-tested phase of this research. 
    The serious game developed by this research didn't show efficacy in the performed test. The pedagogical concepts of the developed game are similar to the pedagogical concepts of others on the same theme, which they show statistical effectiveness.Further studies should be conducted in order to identify the causes that culminated in the achieved result. This research believes that with due retouch, the game has the potential to assume a key role in the prevention of child sexual violence in Brazil.
    

    %Child sexual violence is a public health problem that affects both the victim and society. In this sense, games for the prevention of child sexual violence appear with the aim of providing a playful educational approach to the subject for children. This research introduces an educational game focused on teaching-learning the prevention of sexual violence. The game developed follows pre-established internationally precepts for the education of minors on sexual issues. Among the teachings are concepts associated with the human body and bodily privacy, definition of rights and duties, personal safety and denunciation strategies, and precautions to be taken on the internet. The game is aimed at children from five to eight years old. Its validation is through the Child Abuse Knowledge Questionnaire (CKAQ). The questionnaire is submitted to an experimental group and to a control group in order to investigate significant differences between the groups (95\% confidence level). Both the t-test and the Mann-Whitney test did not point to significant differences between the groups, indicating that there is no statistical significance to state that the children had an increase in their knowledge regarding the topic addressed.   
   
  \textbf{Keywords}: Children sexual abuse; Serious game; Prevention. 
 \end{otherlanguage*}
\end{resumo}
