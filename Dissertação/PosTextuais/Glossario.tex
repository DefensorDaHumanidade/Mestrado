% ----------------------------------------------------------
% Glossário
% ----------------------------------------------------------

%Consulte o manual da classe abntex2 para orientações sobre o glossário.

%\glossary




% ----------------------------------------------------------
% Glossário (Formatado Manualmente)
% ----------------------------------------------------------

\chapter*{GLOSSÁRIO}
\addcontentsline{toc}{chapter}{GLOSSÁRIO}

{ \setlength{\parindent}{0pt} % ambiente sem indentação

%\textbf{Ardósia}: Rocha metamórfica sílico-argilosa formada pela transformação da argila sob pressão e temperatura, endurecida em finas lamelas.

%\textbf{Arenito}: rocha sedimentária de origem detrítica formada de grãos agregados por um cimento natural silicoso, calcário ou ferruginoso que comunica ao conjunto em geral qualidades de dureza e compactação.

%\textbf{Feldspato}: grupo de silicatos de sódio, potássio, cálcio ou outros elementos que compreende dois subgrupos, os feldspatos alcalinos e os plagioclásios.

\textbf{Abuso}: Define um comportamento inaceitável que traz prejuízos muitas vezes irreparáveis à vítima (violência sexual). A definição utilizada neste documento não remete ao uso tradicional associado a palavra abuso (\textit{e.g.} "abusei da comida").

\textbf{Curto Prazo}: Define um período de tempo inferior a 3 (três) anos. 

\textbf{Médio Prazo}: Define um intervalo de tempo entre 3 (três) e 10 (dez) anos. 

\textbf{Longo Prazo}: Define um período de tempo superior a 10 (dez) anos. 

%\textbf{Amostra Qualificada}: ...

\textbf{Nativo Digital}: Termo utilizado para descrever os indivíduos que tiveram contato direto com as tecnologias da era digital desde sua infância. 

\textbf{Imigrante Digital}: Termo utilizado para descrever os indivíduos cujo as tecnologias da era digital lhe foram introduzidas após sua infância. 

\textbf{Primeira Infância}: Define o período que vai do nascimento de um indivíduo até os seus seis anos de idade completos.

\textbf{Deep web}: Termo utilizado para descrever uma parte da \textit{internet} que não pode ser acessada pelos protocolos de rede tradicionais e cujo seu conteúdo não encontra-se indexado por mecanismos de busca. 

%\textbf{Perspectiva Axonométria}: Termo utilizado para as projeções de imagens que estejam em sentido ortogonal ou oblíquo em relação ao observador.

\textbf{Toques Bons}: Termo utilizado ao público infantil para se referir aos tipos de toques aceitáveis na sociedade. 

\textbf{Toques Ruins}: Termo utilizado ao público infantil para se referir aos tipos de toques que trazem mágoa para um determinado indivíduo. 

\textbf{Efeito Hawthorne}: Termo utilizado para definir os indivíduos que mudam ativamente seu comportamento quando sabem que estão sendo observados e monitorados.

\textbf{Falácia da Falsa Causalidade}: Termo utilizado para definir um tipo de falácia que estabeleça uma relação de causa e efeito entre dois elementos quando, na verdade, não existe esse tipo de relação entre elas.



} % fim ambiente sem indentação


