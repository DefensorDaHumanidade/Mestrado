\chapter{Avaliação}\label{ch:Avaliacao}

No âmbito científico, a avaliação de um determinado objeto de estudo busca, de modo geral, elucidar seu real impacto e influências sobre um determinando ambiente. O presente Capítulo apresenta os principais passos tomados para a avaliação do \acf{JS} desenvolvido pelo atual trabalho. Cada etapa influente para a efetiva avaliação do jogo desenvolvido é descrita nas demais seções. Desde modo, a \autoref{sec:Preparativos} descreve sobre os principais preparativos para a execução da pesquisa, a \autoref{sec:seg} fala sobre a segmentação da amostra, a \autoref{sec:pretes} apresenta os resultados alcançados na etapa de pré-teste, a \autoref{sec:tes} os resultados do teste, a \autoref{sec:postes} os resultados do pós-teste, a \autoref{sec:apreciar} descreve os resultados obtidos na etapa de apreciação e a \autoref{sec:compilar} compila todos os resultados, comparando-os ao final com demais trabalhos na área. 

\section{Preparativos}\label{sec:Preparativos}

A corrente pesquisa realiza a avaliação do jogo \textbf{Infância Segura}, de modo a verificar se o jogo em si é capaz de comprir com seus preceitos pré-estabelecidos. O principal preceito do jogo consiste na ideia de que o jogo é capaz de instruir crianças entre 5 (cinco) e 8 (oito) anos a reconhecerem eventos praticados (ou tentados) de violência sexual infantil. Para tal, se faz indispensável a busca por uma amostra de crianças (dentro da faixa etária estabelecida). O conselheiro tutelar Willians Odia e a assistente social Daniella Maragno prestaram seus conhecimentos para a atual pesquisa, elencando possível cenários de atuação. O intercâmbio de ideias levou a Escola Municipal Pauline Parucker (Joinville/\ac{SC}). Após algumas reuniões, a escola prestou seu parecer favorável, servindo de cenário para a execução da presente pesquisa. A diretora Rafaella de Sá Moreira Botelho e a supervisora Angela Marques de Liz Souza, são as principais agentes envolvidas no processo; processo esse, firmado oficialmente pela Declaração de Ciência e Concordância das Instituições Envolvidas (\autoref{chap:DIE}). 

A Escola Municipal Pauline Parucker, se dispos a ceder um total 112 (cento e doze) crianças para a presente pesquisa, dividas em 4 (quatro) turmas: 2º Ano C, 2º Ano D, 3º Ano C, 3º Ano D. As crianças das turmas citadas foram convidados a participar da pesquisa. Para as crianças interessadas em participar da pesquisa foram entregues dois termos, o Termo de Assentimento (\autoref{chap:TA}) e uma versão resumida do Termo de Consentimento Livre e Esclarecido (\autoref{chap:curto}). A versão resumida do Termo de Consentimento Livre e Esclarecido surge de modo a reduzir a quantidade de documentos físicos necessários, sem reduzir de fato, seus conteúdos, isso pois, a versão resumida apresenta um endereço que eletrônico que leva ao termo na integra. Após um período de duas semanas, 33 (trinta e três) documentos retornaram: 8 (oito) do 2º Ano C, 12 (doze) do 2º Ano D, 10 (dez) do 3º Ano C e 3 (três) do 3º Ano D. Salienta-se que durante esse período, um vídeo explicativo sobre a pesquisa foi enviando pela escola (via \textit{WhatsApp}) aos guardiões legais das crianças. Por fim, enfatiza-se que todos os termos e protocolos foram publicados na Plataforma Brasil, os quais foram validados e aprovados pelo Comitê de Ética, sob o \ac{CAAE} nº 43602921.2.0000.0118.

%As crianças, em suas salas de aula e em horário escolar, são apresentadas à pesquisa. Após uma breve apresentação, os menores são convidados a participar da pesquisa. Para as crianças interessadas em participar da pesquisa são entregues duas vias de dois termos. Os termos entregues são o Termo de Assentimento e o Termo de Consentimento Livre e Esclarecido. É requisitado que as crianças apresentem tais termos aos seus guardiões legais, devendo retornar uma das vias de cada um dos termos para a escola. Só participam da corrente pesquisa as crianças com toda a documentação legal devidamente atestada. Visando garantir a integridade das assinaturas e fugir de falsificações cada assinatura é comparada ao documento de matrícula escolar assinado na escola presencialmente pelos pais/responsáveis de cada criança. Todos os termos e declarações do presente estudo são impressos em duas vias. A presente pesquisa se compromete a manter guardada uma das vias por um período de 5 (cinco) anos, assim como todos os registros e resultados alcançados. O término da coleta de toda a documentação legal demarca o início do processo de segmentação da corrente pesquisa. 


\section{Segmentação}\label{sec:seg}

Ao total, 33 (trinta e três) crianças apresentaram toda a documentação necessária para sua participação na corrente pesquisa. Uma vez estabelecido a quantidade de indivíduo aptos a participar no correte estudo, iniciou-se o processo de segmentação da amostra. 

A amostra de indivíduos foi segmentada em dois grupos: Grupo Controle e Grupo Experimental. A segementação resultou em um grupo controle composto por 18 (dezoito) crianças e um grupo experimental composto por 15 (quize) crianças. A disposição detalhada dos grupos ficou da seguinte maneira: grupo controle com 8 (oito) crianças do 2º Ano C e 10 (dez) crianças do 3º Ano C e grupo experimental com 12 (doze) crianças do 2º Ano D e 3 (três) crianças do 3º Ano D.


\section{Pré-teste}\label{sec:pretes}

A etapa de pré-teste do atual trabalho foi realizada no dia 19 (dezenove) de outubro de 2021 às 13h30 (hora local). A etapa de pré-teste surge na presente pesquisa com o objetivo de identificar, a existência ou não, de diferenças significas entre o grupo controle e grupo experimental, no que diz respeito aos seus conhecimentos sobre abuso infantil. A medição dos conhecimentos das crianças é feita com auxílio do instrumento avaliativo \acf{CKAQ} adaptado ao português (\autoref{chap:traduzido}). 

A etapa de pré-teste ocorreu em dois momentos, inicialmente com as turmas do 3º (terceiro) Ano na biblioteca da escola e posteriomente com as turmas do 2º (segundo) Ano em uma sala de aula tradicional. O processo do pré-teste foi inteiramente realizado pelo presente autor com acompanhamento da supervisora Angela Marques de Liz Souza, sendo ela a responsável pela separação e remanegamento das turmas, conforme o regimento escolar. À todas as crianças foi entregue uma cópia impressa do \ac{CKAQ} adaptado ao português. Após a entrega, breves instruções sobre seu conteúdo e sua forma de preenchimento foram passadas (\autoref{chap:teste}). %(semelhante ao \autoref{chap:teste}). Qualquer dúvida poderia ser respondia com a criança levantando a mão.

O questionário foi lindo em voz alta as crianças. O número da questão era lido, juntamente com sua figura representativa para situar as crianças sobre a pergunta corrente do questionário que deveria ser marcada. Uma questão por vez era lida e relida, fornecendo em seguida, uma janela de tempo para as crianças deliberarem e responderem a questão. Foram realizados gestos de positivo e negativo com as mãos, nos nomentos que as crianças eram indagadas se concordavam ou se discordavam (se elas achavam verdade ou se elas achavam mentira) a última questão lida.

O presente estudo não se dispôs a ceder material para o preenchimento do questionário (como lápis e borracha). As crianças participantes, utilizaram seus próprios materiais escolares para assinalar as questões do questionário. O processo como um todo contou com a participação de 31 (trinta e uma)  crianças e durou uma hora e trinta minutos, finalizando às 15h (hora local). Cada momento de aplicação teve duração de quarenta e cinco minutos, com duas crianças faltante do grupo controle (uma de cada turma). Ao término, as crianças retornaram a sua agenda escolar sem demais prejuízos. Por fim, os questionários foram recolhidos e levados para análise.

A análise dos dados do questionário visa compreender melhor o conhecimento de ambos os grupos no que tange os assuntos ministrados pelo questionário em si. Dentre as informações que podem ser coletados por essa análise, está a taxa de acerto por questão. A taxa de acerto por questão do grupo controle pode ser obeservada na \autoref{fig:barrasCon}.

\begin{figure}[htb]

    \caption{\label{fig:barrasCon}Gráfico de Barras da taxa de acerto por questão no pré-teste (grupo controle).}
    \includegraphics[width=\linewidth]{./Visuais/Notas4.pdf}
    \legend{Fonte: Elaborada pelo autor (2021).}
  
\end{figure}

A \autoref{fig:barrasCon} ilustra a taxa de acerto por questão (grupo controle). O eixo das abscissas representa cada uma das questões do questionário, sendo o número 1 (um) a primeira questão do questionário e o número 33 (trinta e três) a última questão do questionário. O eixo das ordenadas representa a taxa de acerto, indo de 0 (zero) a 100 (cem). O número 0 (zero) significa que nenhum indivíduo acertou aquela questão, o número 100 (cem) significa que aquela questão foi acertado por todos os indivíduos. No presente caso, tanto a questão 4 (quatro), quanto a questão 31 (trinta e um), tiveram uma taxa de acerto de 100\% para o grupo controle. O mesmo não ocorre com o grupo experimental como pode ser observado na \autoref{fig:barrasExp}.

\begin{figure}[htb]

    \caption{\label{fig:barrasExp}Gráfico de Barras da taxa de acerto por questão no pré-teste (grupo experimental).}
    \includegraphics[width=\linewidth]{./Visuais/Notas3.pdf}
    \legend{Fonte: Elaborada pelo autor (2021).}
  
\end{figure}

A \autoref{fig:barrasExp} apresenta a taxa de acerto por questão (grupo experimental). Em comparação ao grupo controle (\autoref{fig:barrasCon}) o grupo experimental teve um desempenho ligeiramente inferior. Questões em branco ou rasuradas (de difícil identificação) foram consideradas erradas, ao total 26 (vinte e seis) questões estava rasuradas ou haviam sido deixadas em branco. Para fornecerem um panorama melhor da menor nota, maior nota e a média de cada grupo, produziu-se um diagrama de caixa estreita (\autoref{fig:caixapre}).

\begin{wrapfigure}[17]{r}{10.0cm}%pulando 17 linhas
    \vspace{-4pt}
    \caption{\label{fig:caixapre}Diagrama Caixa Estreita das notas (pré-teste).}
    \vspace{8pt}
    \includegraphics[width=\linewidth]{./Visuais/CaixaEstreitaEnfeitado.pdf}
    \legend{Fonte: Elaborada pelo autor (2021).}
\end{wrapfigure}

Um diagrama caixa estreita mostra a distribuição dos dados em quartis, realçando a média e as exceções. No caso do presente estudo o grupo controle acertou em média 19,625 ($\sigma$ = 3,18) questões (59,47\%, $\sigma$ = 9,64), enquanto o grupo experimental acertou em média 17,467 ($\sigma$ = 3,96) questões (52,93\%, $\sigma$ = 12), valores representados pela marcação em $\times$ na \autoref{fig:caixapre}. A maior quantidade de acertos foi de 26 (vinte e seis) questões no grupo controle e 24 (vinte e quatro) questões no grupo experimental. Ambos os grupos tiveram uma menor quantidade de acertos de 13 (treze) questões.

Para a devida condução da atual pesquisa é preciso garantir que as amostram sejam equivalentes entre si. Para averiguar o grau de equivalência entre duas amostras a estatística faz uso do Teste-T. Contudo, para se utilizar o Teste-T, é preciso determinar a priori, se a variância entre os grupos é igual ou diferente. Após os cálculos, os resultados indicaram que as  variâncias do grupo controle e grupo experimental são equivalentes ($\alpha$ = 0,05). Deste modo, o Teste-T foi realizado assumindo variância iguais entre as amostras, concluíndo ao final que as amostras são de fato, equivalentes ($\alpha$ = 0,05). A \autoref{fig:normal} ilustra a equivalência entre as amostras. 

\begin{figure}[htb]
    \centering
    \caption{\label{fig:normal}Distribuição das notas atingidas no pré-teste}
    \includegraphics[width=\linewidth]{./Visuais/Graficos1.pdf}
    \legend{Fonte: Elaborada pelo autor (2021).}
  
\end{figure}


\section{Teste}\label{sec:tes}

A etapa de teste do atual trabalho ocorreu em três momentos distintos; dia 3 (três), dia 4 (quatro) e dia 5 (cinco) de novembro de 2021. No dia 3 (três) foram realizados os preparativos para a instalação do jogo nos \textit{tablets} da escola (modelo \textit{Multilaser} M10A). O processo se iniciou às 14h e terminou às 16h15 (hora local). O processo de instalação do jogo nos \textit{tablets} se deu na biblioteca da escola e contou com a participação da professora Carla Diacui Medeiros Berkenbrock, a suprevisora Angela Marques de Liz Souza e o presente autor desta dissertação. Ao total o jogo foi instalado em 30 (trinta) \textit{tablets}. Foi realizada a instalação do jogo em mais \textit{tablets} do que o necessário, visando contornar demais empecilhos como problemas ou falta de energia em algum determinado dispositivo durante a etapa de teste (apenas um \textit{tablets} foi trocado durante os experimentos). Ao final, todos os \textit{tablets} foram colocados em um gabinete de recarga.

Nos dias 4 (quatro) e 5 (cinco) de novembro de 2021 a presente pesquisa conduziu sua etapa de teste com crianças (na bilioteca da escola). O processo consistiu em submeter aos participantes (crianças), o objeto estudado (jogo). No dia 4 (quatro) as crianças participantes foram instruidas a jogar o jogo \textbf{Infância Segura}, mais especificamente as fases sobre \textbf{Direitos} (\autoref{subsec:2}) e \textbf{Denúncias} (\autoref{subsec:3}). No dia 5 (cinco) as crianças foram instruidas a jogar as demais fases, sobre \textbf{Anatomia} (\autoref{subsec:1}) e \textbf{Redes Sociais} (\autoref{subsec:4}). Essa separação das fases foi uma sugestão da suprevisora Angela. É importante salientar que a etapa de teste foi conduzida inteiramente e apenas com as crianças do grupo experimental, com uma criança do 3º Ano D faltante no dia 5 (cinco). Em ambos os dias o processo se iniciou às 13h40.

No dia 4 (quatro) e no dia 5 (cinco) os \textit{tablets} foram retirados do gabinete de recarga e distribuídos sobre as mesas da biblioteca da escola. As crianças foram entrando, duas a duas, na biblioteca da escola, sendo instruídas a sentarem na frente do \textit{tablet} que contivesse uma etiqueta com seu nome escrito (no último dia a posição dos \textit{tablets} foi trocada). Em ambos os dias as crianças tiveram contato com o jogo por quarenta e cinco minutos, totalizando ao total uma hora e trinta minutos de atividades (com início às 14h15 e término às 15h). O processo foi inteiramente conduzido pelo presente autor desta dissertação, com auxílio da suprevisora Angela (todos os dias). Em separado, no 4 (quatro) a professora Carla Diacui Medeiros Berkenbrock e no dia 5 (cinco) a aposentada Rocilda Cordeiro Mendonça, prestaram sua atenção e auxílio a pesquisa. Em específico a suprevisora Angela foi a responsável pelo remanejamento dos estudantes. %Em específico a suprevisora Angela foi a responsável por remanejar os estudantes das suas salas à biblioteca (e vice-versa).

No primeiro dia, as crianças foram apresentadas ao jogo com auxílio de um projetor. Neste momento as crianças foram instruídas a se identificarem no jogo e a informarem seu gênero. As crianças também foram instruídas a como alterar o volume do jogo e a como intercambiar entre as fases. No segundo dia, com auxílio do mesmo projetor, realizou-se uma revisão do terceiro minijogo da fase sobre \textbf{Direitos} e do primeiro minijogo da fase sobre \textbf{Redes Sociais}. Tais minijogos foram escolhidos para uma revisão, uma vez observado que os conteúdos contidos nestes minijogos coincidiam com os conteúdos das questões com menores taxas de acerto para o grupo experimental na etapa de pré-teste (\autoref{sec:pretes}).

No segundo dia, as crianças (em geral) apresentaram muita dificuldade para concluir o primeiro e o segundo minijogo da fase sobre \textbf{Anatomia}. Em virtude dessa dificuldade, instruções mais detalhadas de cada minijogos foram passadas com auxílio de um projetor. Por fim, a taxa de conclusão do jogo (quatro fases) ficou em 89,63\%. Ao todo 7 (sete) crianças deixaram de concluir uma ou mais fases no jogo. Os minijogos da etapa de \textbf{Direitos} e \textbf{Redes Sociais} foram os minijogos com melhor desempenho entre os jogadores. Seguidos pelos minijogos da etapa de \textbf{Denúncias} e pelos minijogos da etapa de \textbf{Anatomia}, com os piores desempenhos. 

No segundo dia, a rede sem fio da escola apresentou instabilidade. Tal instabilidade impossibilitou que os dados dos estudantes (gerados durante o jogo) fossem enviados aos servidores da aplicação. No entanto, os dados gerados pelos jogadores durante a etapa de teste, além de estarem programados para serem enviandos a um banco de dados, também estavam programados para armazanar todo conjunto de ações dos seus jogadores localmente (nos \textit{tablets}). Com isso, após o fim dos experimentos no segundo dia, o conteúdo de cada jogador foi extraido dos \textit{tablets} (o processo levou em torno de uma hora). Por fim, a etiqueta dos tablet foi removida, os \textit{tablets} foram desligados e colocados em um gabinete de recarga. Ao total, 3.315 ações foram computadas nos minijogos, dando em média, 221,83 ações por jogador.

No decorrer da etapa de teste com as crianças, nenhum dos envolvidos manifestou qualquer quadro de angústia, nervossímos, ansiendade ou constrangimento para com os conteúdos abordados pelo jogo. Destaca-se apenas que, uma das crianças, apresentou sinais de preocupação, ao sentir que estava demorando mais do que os seus outros colegas para concluir os minijogos. Também destaca-se que, algumas crianças, tiveram dificuldades em compreender os diálogos do jogo, pedindo muitas vezes, para que os diálogos fossem lidos para elas. O fato de muitas crianças não estarem plenamente alfabetizadas, acabou por atrapalhar a absorção do conteúdo pelas crianças. Isso pois, por mais que o jogo estive localizado em português, a execução conjuta dos \textit{tablets} em um mesmo ambiente acabou gerando um certo ruído sonoro no ambiente, dificultando os áudios de serem ouvidos. As crianças levantavam suas mãos quando necessitavam de alguma ajuda ou auxílio, quando precisavam ir ao banheiro ou quando gostariam de informar que concluiram com todo o conteúdo. As crianças que manigestaram terem concluido todo o jogo foram liberadas a jogar qualquer minijogo de seu interesse (até o término das atividades). 

O jogo em si apresentou algumas falhas de desenvolvimento durante a etapa de teste, falhas tanto técnicas quanto conceituais. Algumas ideias para o aperfeiçoamento do jogo e sugestões de melhorias foram manifestadas também durante a etapa de teste. Tanto as falhas constatadas, quanto as sugestões de melhoramento do jogo são tratadas com maiores detalhes no \autoref{ch:Conclusao} do presente trabalho acadêmico.



\section{Pós-teste}\label{sec:postes}


Praprado na escola 22 (vinte e cinco) de novembro de 2021 Praprado na escola 22 (vinte e cinco) de novembro de 2021Praprado na escola 22 (vinte e cinco) de novembro de 2021Praprado na escola 22 (vinte e cinco) de novembro de 2021Praprado na escola 22 (vinte e cinco) de novembro de 2021Praprado na escola 22 (vinte e cinco) de novembro de 2021Praprado na escola 22 (vinte e cinco) de novembro de 2021Praprado na escola 22 (vinte e cinco) de novembro de 2021Praprado na escola 22 (vinte e cinco) de novembro de 2021

\section{Apreciação}\label{sec:apreciar}

Praprado na escola 25 (vinte e cinco) de novembro de 2021

\section{Compilação}\label{sec:compilar}

\subsection{Meus}\label{subsec:meus}


112 crianças (Turmas C e D, 2 e 3)
2C = 8 crianças
2D = 12 crianças
3C = 10 crianças
3D = 3 crianças


\subsection{Comparativo}\label{subsec:outros}

%\section{Considerações Finais}\label{sec:compilar}


*Resultados Finais

*Conclusão

\newcolumntype{P}[1]{>{\centering\arraybackslash}p{#1}}
\newcolumntype{M}[1]{>{\centering\arraybackslash}m{#1}}
\begin{table}[htb]
\footnotesize
\renewcommand{\arraystretch}{1.5} %espaço entre as linhas
\caption[Trabalhos Relacionados.]{Trabalhos Relacionados.}
\label{tab-nivinv}
\centering
\begin{tabular}{p{4.2cm}p{2.0cm}M{2.2cm}M{1.5cm}M{2.0cm}M{2.1cm}}
  \toprule
   \textbf{Jogo (Ano)} & \textbf{Idioma}  & \textbf{Público-alvo} & \textbf{Amostra} & \multicolumn{2}{c}{\textbf{Validação (pós-teste)}}  \\
   \cline{5-6}
     &  &  &  & Grupo Controle & Grupo Experimental\\
   \midrule
    \textit{Being Safety Smart} (2009)  & Inglês          & 6-8 anos    & 76    & \textcolor[rgb]{0.9,0,0}{\textbf{69\%}}      & \textcolor[rgb]{0,0.6,0}{\textbf{90\%}} \\
    \hline
    \textit{Orbit Rescue} (2012)        & Inglês          & 8-10 anos   & 139    & \textcolor[rgb]{0.9,0,0}{\textbf{75\%}}      & \textcolor[rgb]{0,0.6,0}{\textbf{93\%}} \\
    \hline
    \textit{Cool and Safe} (2013)       & $\tfrac{Alem\tilde{a}o/Franc\hat{e}s}{Ingl\hat{e}s/Espanhol}$  & 7-12 anos   & 286   & \textcolor[rgb]{0.9,0,0}{\textbf{61\%}}      & \textcolor[rgb]{0,0.6,0}{\textbf{79\%}} \\
    \hline
    \textit{Infância Segura} (2021)     & $\tfrac{Portugu\hat{e}s}{Espanhol/Ingl\hat{e}s}$       & 5-8 anos    & 33      &   \textcolor[rgb]{0.9,0,0}{\textbf{-59\%}}       &   - \\
    \bottomrule
\end{tabular}
\legend{Fonte: os autores}
\end{table}







A atual pesquisa submente o instrumento avaliativo \ac{CKAQ} para sua amostra de participantes (grupo controle e grupo experimental). A submissão do \ac{CKAQ} ocorre em duas etapas distintas da presente pesquisa, inicialmente na etapa de pré-teste e posteriormente na etapa de pós-teste. O instrumento avaliativo é entregue de maneira impressa e administrado verbalmente em sala de aula e em horário escolar para todos os participantes válidos de uma turma. Participantes inválidos (\textit{e.g.} crianças com participação não aprovada por seus guardiões legais) são direcionados para um atividade escolar sob responsabilidade da escola. Após a administração do \ac{CKAQ}, suas cópias impressas com as respostas dos participantes são colhidas e embaralhadas. Tal coleta encerra a etapa de pré-teste e demarca o início da etapa de teste da presente pesquisa.

%Tal versão é submetida ao grupo controle e ao grupo experimental em momentos distintos de maneira coletiva. O questionário é adminitrado verbalmente e as crianças devem responde-lo em suas folhas. Cada criança tem um indetificador. O questionário tem previsão de ser respondido em 15 minutos. Essa etapa consiste em colher a Aprendizagem (dado quantitativo) sobre  a temática de ambos os grupos. O questionário pode ser administrado por um dos pesquisadores ou por outro responsável optado pela escola. Ao término do questionário as folhas serão colhidas de cada participantes para análise. 


%Os resultados alcançados com o grupo estudado se demonstrar bem sólidos e robustos. Isso pois, a presente pesquisa se utiliza do Teste-t com um grau de confiança de 95\%. A alta taxa de confiança estatística do presente estudo, abre margem para uma defesa sólida e robusta de seus resultados. 

TESTES TESTES.....

Uma turma com 112 (cento e doze crianças) foram seleciondas para participantarem dessa pesquisa da Escola Municipal Pauline Parucker. Todos os protoclos foram segufios....









