\chapter{Avaliação}\label{ch:Avaliacao}


As crianças, em suas salas de aula e em horário escolar, são apresentadas à pesquisa. Após uma breve apresentação, os menores são convidados a participar da pesquisa. Para as crianças interessadas em participar da pesquisa são entregues duas vias de dois termos. Os termos entregues são o Termo de Assentimento e o Termo de Consentimento Livre e Esclarecido. É requisitado que as crianças apresentem tais termos aos seus guardiões legais, devendo retornar uma das vias de cada um dos termos para a escola. Só participam da corrente pesquisa as crianças com toda a documentação legal devidamente atestada. Visando garantir a integridade das assinaturas e fugir de falsificações cada assinatura é comparada ao documento de matrícula escolar assinado na escola presencialmente pelos pais/responsáveis de cada criança. Todos os termos e declarações do presente estudo são impressos em duas vias. A presente pesquisa se compromete a manter guardada uma das vias por um período de 5 (cinco) anos, assim como todos os registros e resultados alcançados. O término da coleta de toda a documentação legal demarca o início do processo de segmentação da corrente pesquisa. 


A atual pesquisa submente o instrumento avaliativo \ac{CKAQ} para sua amostra de participantes (grupo controle e grupo experimental). A submissão do \ac{CKAQ} ocorre em duas etapas distintas da presente pesquisa, inicialmente na etapa de pré-teste e posteriormente na etapa de pós-teste. O instrumento avaliativo é entregue de maneira impressa e administrado verbalmente em sala de aula e em horário escolar para todos os participantes válidos de uma turma. Participantes inválidos (\textit{e.g.} crianças com participação não aprovada por seus guardiões legais) são direcionados para um atividade escolar sob responsabilidade da escola. Após a administração do \ac{CKAQ}, suas cópias impressas com as respostas dos participantes são colhidas e embaralhadas. Tal coleta encerra a etapa de pré-teste e demarca o início da etapa de teste da presente pesquisa.

%Tal versão é submetida ao grupo controle e ao grupo experimental em momentos distintos de maneira coletiva. O questionário é adminitrado verbalmente e as crianças devem responde-lo em suas folhas. Cada criança tem um indetificador. O questionário tem previsão de ser respondido em 15 minutos. Essa etapa consiste em colher a Aprendizagem (dado quantitativo) sobre  a temática de ambos os grupos. O questionário pode ser administrado por um dos pesquisadores ou por outro responsável optado pela escola. Ao término do questionário as folhas serão colhidas de cada participantes para análise. 


TESTES TESTES.....

Uma turma com 112 (cento e doze crianças) foram seleciondas para participantarem dessa pesquisa da Escola Municipal Pauline Parucker. Todos os protoclos foram segufios....