\chapter{Considerações Finais}\label{ch:Conclusao}

O \acf{ASI} é um grave problema de predominância global que assola milhares de crianças todos os anos. Os danos da violência sexual infantil são largamente conhecidos e documentados na literatura médica da área, os quais podem acompanhar a criança violentada durante a vida inteira. Em resposta, inúmeras estratégias voltadas ao enfretamento deste mal surgiram (\autoref{ch:Relacionados}). Estratégias focadas na capacitação de crianças estão entre as mais promissoras. Isso ocorre, pois os agressores sexuais preferem agir sobre crianças com menores chances de manifestarem resistência aos seus abusos, por tal razão que a instrução e capacitação de crianças é uma ação capaz de coibir a ocorrência de tal violência, implicando assim, na redução dos crimes.

A capacitação de crianças é uma estratégia promissora no combate à violência sexual infantil. O aperfeiçoamento da segurança pessoal por meio de programas de capacitação e prevenção é uma atitude capaz de evitar a incidência de episódios de abuso. Diante deste fato, a presente pesquisa se prestou a desenvolver um programa de educação voltado a coibir o abuso sexual de crianças. O programa desenvolvido baseia-se na ideia de \acfp{JS}. Estratégias baseadas em jogos didáticos vêm a agregar e facilitar o aprendizado infantil, principalmente em temas extremamente sensíveis e delicados, como na temática da violência sexual. Uma abordagem baseada em jogos fornece um meio de aprendizagem poderoso, promovendo uma abordagem educacional divertida e envolvente para a prevenção da violência infantil. A utilização de jogos, permite que alunos possam aprender através da vivência de situações simuladas, sem ter que passar por elas efetivamente. Por tal razão que o presente trabalho acadêmico se fundamentou na ideia proposta por \citeonline{diocesano2018infancia}. 

\citeonline{diocesano2018infancia} propõe um jogo para combater a violência sexual infantil. O jogo proposto teve seus ensinamentos revisados por especialistas na área e encontra-se em total conformidade com orientações técnicas internacionais de educação em sexualidade. O código-fonte do jogo desenvolvido pela presente pesquisa é de propriedade da \ac{UDESC}, o qual encontra-se publicado no \autoref{chap:codigo}. %jogo é de código aberto e de licença livre, permitindo sua adaptação e expansão para outros idiomas.

A corrente pesquisa realiza a avaliação do jogo desenvolvido, de modo a verificar se o jogo em si é capaz de cumprir com seus preceitos pré-estabelecidos. O processo de avaliação do jogo foi realizado com uma amostra de 33 (trinta e três crianças), segmentadas em dois grupos: Grupo Controle e Grupo Experimental. Os resultados iniciais apontaram para uma equivalência estatística entre os grupos, tanto em suas médias, quanto em suas medianas. Tal resultado, permitiu que o jogo desenvolvido fosse exposto ao grupo experimental, de modo a constatar a influência do jogo sobre a amostra. Como resultado final, o jogo desenvolvido não apresentou qualquer interferência que pudesse ser identificada estatisticamente (grau de confiança de 95\%). Em outras palavras, ambos os grupos estudados apresentaram graus de conhecimento semelhantes na etapa de pré e pós-teste da pesquisa, no que diz respeito aos conhecimentos medidos pelo instrumento avaliativo \ac{CKAQ}.

%O \acf{JS} desenvolvido pela presente pesquisa não aparenta ter provocado qualquer resposta cognitiva de conhecimento ao grupo estudado. A amostra da corrente pesquisa não obedece uma distribuição totalmente aleatória, isso pois indivíduos envolvidos no processo de validação do jogo pretencem todos um mesmo ambiente e contexto semelhantes. Salienta-se então, que os resultados e achados desta pesquisa se demonstram válidos apenas para o grupo estudado, nada se pode afirmar sobre a estrapolação dos resultados desta pesquisa para a população em geral. É preciso que pesquisa futuras sejam conduzidas de modo a abrager tais critérios, possibilitando o mapeamento de variáveis de modo a extrapolar seus resultados para a população geral. 
 
O \acf{JS} desenvolvido pela presente pesquisa não aparenta ter provocado qualquer resposta cognitiva de conhecimento ao grupo estudado. Uma das possíveis causas deste resultado, diz respeito ao nível de maturidade dos grupos. Conforme outras pesquisas na área apontam (\autoref{ssec:TR}), crianças mais velhas apresentam um desempenho ligeiramente superior quando submetidas ao instrumento avaliativo \ac{CKAQ}, em comparação a crianças mais novas. O grupo controle da corrente pesquisa manifesta uma idade média de 8,19 anos, já o grupo experimental manifesta uma idade média de 7,84 anos. Tal diferença entre as idades, pode ser um fator de influência sobre os resultados. Todavia, salienta-se que mesmo o grupo experimental sendo ligeiramente mais novo que o grupo controle, o grupo experimental obteve na etapa de pós-teste da corrente pesquisa um desempenho mediano ligeiramente  superior (63,63\%), ao grupo controle (62,12\%). Por fim, destaca-se que durante a execução deste trabalho de mestrado não foram manifestados e nem identificados casos suspeitos ou consumados de violência sexual infantil para com os participantes desta pesquisa.  

%É importante deixar claro que a presente pesquisa se prestou apenas mensurar aprendizagem das crianças no que diz respeito a influência de um \acf{JS} sobre seus conhecimentos. A corrente pesquisa não se presta mensurar o comportamento infantil sobre tentativas simuladas de abuso, de modo a identificar se as crianças ampliaram suas habilidades preventias ou não.


%Os programas voltados a educar as menores faixas etárias são os mais criticados, por responsabilizarem demasiadamente a criança \cite{colleen2016advancing}. Nesses programas, existe a preocupação da criança culpar a si mesma após um episódio de violência, por não ter evitado o abuso, possivelmente agravando ainda mais os traumas deixados no menor.



%É importante tomar cuidado com CORRELAÇÃO VS CAUSALIDADE: e.g. Divorce rate in Maine vs Per capita consumption of magarine        Number of films Niclas Cage appeared in vs Female Editors on Harvard Law Review



%Outra questão a ser levada em consideração se relaciona com a responsabilidade imposta as crianças pelos programas de capacitação. Há a preocupação de alguns pesquisadores na área que programas do gênero podem trazer um sentimento de culpa as crianças envolvidas a episódio de abuso, piorando assim o quadro clínico dos menores nestes casos \cite{finkelhor2009prevention}. %“burden of responsibility”

%Almeja-se que os ensinamentos de prevenção a violência sexual sejam incluídos na Base Nacional Comum Curricular (e não apenas conhecimentos de reprodução e demais afins). Deste modo, surge a chance para a inclusão do jogo desenvolvido em salas do ensino fundamental, com o jogo agindo como um agregador e não como um substituto das aulas tradicionais, trazendo assim mais engajamento e ludicidade as aulas de prevenção à violência sexual infantil.
   

\section{Limitações da Pesquisa}\label{sec:limitacoes}

O atual trabalho acadêmico obteve algumas limitações durante sua realização. A pandemia causada pelo coronavírus SARS-CoV2, causador da \ac{SRAG} atrapalhou a execução dos experimentos da corrente pesquisa (\autoref{ch:Avaliacao}). Fones de ouvido não foram utilizados durante a execução dos experimentos com o jogo, buscando reduzir a quantidade de utensílios de contato. A não utilização de fones de ouvido acabou por gerar um ruído no ambiente que dificultou a absorção dos conteúdos abordados no jogo pelas crianças. Durante a etapa de apreciação da pesquisa (\autoref{sec:apreciar}) as crianças manifestaram inclusive, essa dificuldade em entender e compreender as falas do jogo. Outro fator possivelmente limitante causado pela pandemia, diz respeito a baixa adesão de voluntários no corrente estudo. De 112 (cento e doze) crianças, somente 33 (trinta e três) apresentaram documentação necessária para participar da pesquisa, uma taxa de adesão de 29,46\%. A baixa adesão de crianças na corrente pesquisa pode ser resultado de uma medida cautelar, dos pais ou responsáveis, que optaram por limitar as atividades de suas crianças ao mínimo necessário. 

Outro fator limitante remete as interrupções ocorridas durante a etapa de pré e pós-teste da corrente pesquisa (\autoref{ch:Avaliacao}). O instrumento avaliativo \ac{CKAQ} utilizada na etapa de pré e pós-teste teve sua aplicação interrompida brevemente por terceiros. A presença breve de terceiros pode ter distraído algumas crianças, o que poderia explicar a não marcação de alguns itens no questionário. %Algumas crianças foram no banheiro (na etapa de testes com o jogo).

Por fim, o último fator limitante a ser citado diz respeito ao não envolvimento contínuo de todas as crianças. Nenhuma das etapas da presente pesquisa contou com a participação ativa e completa da amostra de crianças. Ou seja, durante a execução do processo avaliativo, o processo como um todo não teve em nenhum momento a participação das 33 (trinta e três) crianças envolvidas nessa pesquisa. 

\section{Trabalhos Futuros}\label{sec:futuros}

Os resultados e achados desta pesquisa não são válidos para toda a população em geral. Todas as descobertas envolvendo o desenvolvimento de habilidades preventivas à violência sexual por meio de um jogo educativo são válidas apenas para o grupo testado. As crianças participantes da atual pesquisa não possuem grandes diferenças étnicas ou socioeconômicas. Desta forma, os resultados e achados do atual estudo podem não valer para toda a população em geral. É preciso que pesquisas futuras sejam conduzidas de modo a abranger uma amostra de participantes com maior variabilidade, podendo assim, extrapolar seus resultados para a população geral. Todavia, os resultados alcançados com o grupo estudado se demonstram sólidos e robustos. Isso pois, a presente pesquisa se utiliza do Teste \textit{t} com um grau de confiança de 95\%. A alta taxa de confiança estatística do presente estudo, abre margem para uma defesa sólida e robusta de seus resultados.

Os resultados e dados obtidos durante os testes com o jogo podem ter sido corrompidos. Embora o jogo tenha um sistema de redundância para os dados, gravando-os internamente e enviando-os ao servidor, não há um sistema de verificação de integridade dos dados. Por tal razão, é mais do que recomendado que pesquisas futuras se utilizem de um sistema de verificação, para não haver o risco da perda ou do corrompimento dos dados (\textit{e.g. Checksum}).

%(esse sistema não é tão eficaz como um Checksum, porém se demonstra um )  FAZER CHESSUM NO JOGO

Durante a etapa de testes com o jogo três sugestões de melhoria foram apresentadas pelos professionais envolvidos. A primeira sugestão para um trabalho futuro remete a um sistema de identificação das fases já concluídas. Durante a etapa de testes com o jogo, forneceu-se liberdade para as crianças jogarem as fases do jogo na ordem que melhor lhe agradasse, isso gerou um problema que dificultava identificar quais fases as crianças já haviam concluído, pois muitas crianças também não lembravam. Tal condição poderia justificar o fato do jogo não ter sido concluído totalmente pelas crianças. 

Outra sugestão apresentada por um dos profissionais envolvidos para um trabalho futuro está associada a ideia de aleatorização das perguntas em relação as imagens no que diz respeito aos níveis apresentados na \autoref{subsec:2}. No caso, observou-se durante os experimentos que as crianças, ao rejogarem o jogo, elas selecionavam as figuras que haviam memorizado como corretas, sem dar atenção novamente aos diálogos para reforçar seus conhecimentos. Nesse sentido, surgiu a ideia de aleatorizar as perguntas com as imagens de modo a ter duas perguntas, uma na negativa e outa na positiva. Deste modo, não haveria como os jogadores memorizarem a imagem correta, pois as perguntas para as mesmas imagens poderiam mudar, mudando assim, a resposta final. 

Outra sugestão dada, foi a remoção do sistema de pontuação por tempo. Acredita-se que muitas crianças poderiam ter se apegado a esse sistema, respondendo e realizando as ações no jogo de forma mais rápida e automatizada possível, sem nem prestar atenção nos ensinamentos dados. 

\newpage

Durante a etapa de apreciação (\autoref{sec:apreciar}) algumas crianças manifestaram que consideraram o jogo muito fácil, tal manifestação ajuda a compreender a menor taxa de satisfação observada na questão de número 6 (seis) do questionário \ac{MEEGA} adaptado. Como pesquisa futura recomenda-se a remodelagem do jogo, de modo a melhor balancear seus níveis de dificuldade em cada fase, de modo a deixá-lo mais desafiador aos jogadores. As crianças também manifestaram a inclusão de mais cenários no jogo, ampliando assim a variabilidade dos ambientes. 

Como melhoria no jogo recomenda-se como trabalho futuro, um ajuste nos comandos de toque do jogo. Durante a execução dos testes, vários elementos apresentaram sinais de travamento quando eram selecionados e arrastados. Inclusive um dos \textit{tablets} foi trocado devido a essas condições. Destaca-se que tal observação pode ser resultado do uso de álcool em gel durante os experimentos ou outra variável não identificada. 

% o segundo pelo fato do jogo não manifestar qualquerdesconforto ou mudança inadequada de comportamento aos jogadores

%o, testes emocionais foram realizados, os quais não identificaram quaisquer efeitos colaterais negativos manifestados pelas crianças após a conclusão do jogo, demonstrando que seu conteúdo se encontra adequado a idade ministrada.


%Um jogo que seja rejeitado pelo processo avaliativo deve ter seu desenvolvimento retomado para corrigir eventuais falhas constatadas.

%Nenhuma criança maniestou sinais de violencia sexual durante a pesquisa.



%Algumas crianças tiveram dificuldade em colocar as peças e NÃO COMPREENDERAM O QUE DEVIA SER FEITO NA SEGUNDA FASE DO HOSPITAL. Elas não sabia que deviam marcar as partes intimas, e nem sabia quais partes deveriam marcar (jogo MUITO POUCO INTUITO).



%É importante nenhuma das crianças que ojgaram o apresentam Daltonismo....  A falta de fone foi um fator triste (muito barulho atrapalhando as crianças).




\section{Publicações}\label{sec:publicar}

Este trabalho de mestrado resultou nas seguintes publicações:

\begin{itemize}
    \item FAVA, Alexandre Mendonça; BERKENBROCK, Carla Diacui Medeiros. O Professor como Coordenador em um Jogo para Prevenção da Violência Sexual Infantil. In: \textbf{XV Simpósio Brasileiro de Sistemas Colaborativos}, Porto Alegre, v. 15, 3 out. 2019.
    %https://sol.sbc.org.br/index.php/sbsc_estendido/article/view/8354
    \item FAVA, Alexandre Mendonça; BERKENBROCK, Carla Diacui Medeiros. Um jogo sério como tecnologia educacional para prevenção da violência sexual infantil. In: \textbf{V Colóquio Luso-Brasileiro de Educação}, Joinville, v. 4, 24 mar. 2020. 
    %https://www.revistas.udesc.br/index.php/colbeduca/article/view/17215
    \item FAVA, Alexandre Mendonça.; BERKENBROCK, Carla Diacui Medeiros.; VAHLDICK, Adilson. Avaliação e remodelagem de um jogo sério para prevenção do abuso sexual infantil. \textbf{Revista Brasileira de Computação Aplicada}, v. 13, n. 1, p. 112-124, 5 abr. 2021. ISSN 2176-6649.
    %http://seer.upf.br/index.php/rbca/article/view/10894
\end{itemize}