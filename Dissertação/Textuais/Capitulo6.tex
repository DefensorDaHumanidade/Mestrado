\chapter{Considerações Finais}\label{ch:Conclusao}

%(esse sistema não é tão eficaz como um Checksum, porém se demonstra um )  FAZER CHESSUM NO JOGO

O abuso sexual infantil é um grave problema de predominância global que assola milhares de crianças todos os anos. Os danos da violência sexual infantil são largamente conhecidos e documentados na literatura médica da área, os quais podem acompanhar a criança violentada durante a vida inteira. Em resposta a esse problema, inúmeras estratégias surgiram. A capacitação de crianças é uma estratégia promissora que se destaca entre as demais estratégias. O aperfeiçoamento da segurança pessoal por meio de programas de capacitação é uma atitude capaz de evitar a ocorrência de episódios de abuso. Por meio dos programas de capacitação para crianças o problema da violência sexual é cortado pela raiz, pois os agressores sexuais evitam abordar crianças com maiores probabilidades de recusar e relatar suas abordagens abusivas para as devidas autoridades. 

Em vias de mitigar o problema da violência sexual infantil no Brasil a presente pesquisa almeja desenvolver um programa de capacitação para crianças de cinco a oito anos de idade. O programa em questão possui seus conceitos pedagógicos baseados em orientações técnicas internacionais de educação em sexualidade. A dinâmica do programa assume o caráter de um jogo sério. Uma abordagem baseada em jogos fornece um meio de aprendizagem promovendo uma abordagem educacional divertida e envolvente para a prevenção da violência infantil. A utilização de jogos, permite que alunos possam aprender através da vivência de situações simuladas, sem ter que passar por elas efetivamente.

Em relação aos programas tradicionais para a prevenção da violência sexual infantil, o programa proposto pela presente pesquisa se utiliza de um jogo educacional digital, o qual permite que os menores possam se manifestar em sua plenitude, sem se sentirem intimidados ou acanhados. 
%referencia muller2014child
Além disso, uma estratégia digital ainda permite que os custo de expansão do programa sejam mais reduzidos em comparação a estratégias presenciais. 

A atual pesquisa ainda encontra-se em processo de desenvolvimento. Em virtude de sua temática sensível e da vulnerabilidade do público alvo, se faz necessário que o atual trabalho passe pelos devidos protocolos do Comitê de Ética. Inclusive, em virtude de questões éticas não se faz possível que essa pesquisa seja capaz de mensurar o comportamento infantil a tentativas simuladas de abuso. Por tal razão, os dados a serem obtidos sobre a aprendizagem das crianças no programa não podem ser inferidos para suas atitudes comportamentais em situações de abuso. 

%MOSTRAR AS VARIANCIAS COMO FOI FEITO NOS OUTROS TRABALHOS.





Logo, os resultados e achados desta pesquisa se demonstram válidos apenas para o grupo estudado, nada se pode afirmar sobre a estrapolação dos resultados desta pesquisa para a população em geral. É preciso que pesquisa futuras sejam conduzidas de modo a abrager tais critérios, possibilitando o mapeamento de variáveis de modo a extrapolar seus resultados para a população geral.

Os programas voltados a educar as menores faixas etárias são os mais criticados, por responsabilizarem demasiadamente a criança \cite{colleen2016advancing}. Nesses programas, existe a preocupação da criança culpar a si mesma após um episódio de violência, por não ter evitado o abuso, possivelmente agravando ainda mais os traumas deixados no menor.

%Os resultados e achados desta pesquisa não são válidos para toda a população em geral. Todas as descobertos envolvendo o desenvolvimento de habilidades preventivas a violência sexual por meio de um jogo educativo são válidas apenas para o grupo testado. As crianças participantes da atual pesquisa não possuem grandes diferenças étnicas ou socio-econômicas. Desta forma, os resultados e achados do atual estudo podem não valer para toda a população em geral. É preciso que pesquisa futuras sejam conduzidas de modo a abrager um amostra de participantes com maior variabilidade, podendo assim, extrapolar seus resultados para a população geral. Todavia, os resultados alcançados com o grupo estudado se demonstrar bem sólidos e robustos. Isso pois, a presente pesquisa se utiliza do Teste-t com um grau de confiança de 95\%. A alta taxa de confiança estatística do presente estudo, abre margem para para uma defesa sólida e robusta de seus resultados. 

%É importante tomar cuidado com CORRELAÇÃO VS CAUSALIDADE: e.g. Divorce rate in Maine vs Per capita consumption of magarine        Number of films Niclas Cage appeared in vs Female Editors on Harvard Law Review

\pagebreak

Outra questão a ser levada em consideração se relaciona com a responsabilidade imposta as crianças pelos programas de capacitação. Há a preocupação de alguns pesquisadores na área que programas do gênero podem trazer um sentimento de culpa as crianças envolvidas a episódio de abuso, piorando assim o quadro clínico dos menores nestes casos \cite{finkelhor2009prevention}. %“burden of responsibility”

Almeja-se que os ensinamentos de prevenção a violência sexual sejam incluídos na Base Nacional Comum Curricular (e não apenas conhecimentos de reprodução e demais afins). Deste modo, surge a chance para a inclusão do jogo desenvolvido em salas do ensino fundamental, com o jogo agindo como um agregador e não como um substituto das aulas tradicionais, trazendo assim mais engajamento e ludicidade as aulas de prevenção à violência sexual infantil.
   

O jogo a ser desenvolvido pelo presente trabalho é uma continuação da dissertação do professor Tiago Francisco Andrade Diocesano. %, o qual batizou o jogo de \textit{Infância Segura}. 
O jogo desenvolvido nesta pesquisa é de propriedade da Universidade do Estado de Santa Catarina. Entretanto, o jogo é de código aberto e de licença livre, permitindo sua adaptação e expansão para outros idiomas. Os próximos passos da presente pesquisa são apresentados no Quadro \ref{tabelinha}.

\captionsetup[table]{name=Quadro}
\begin{table}[!htb]
    \centering
    \renewcommand{\arraystretch}{1.5} %espaço entre as linhas
    \caption{\emph{Cronograma de Atividades para o primeiro semestre de 2021}.}\label{tabelinha}
    \vspace{0.2cm}
    \begin{tabular}{|p{8cm}|c|c|c|c|c|c|}
    \hline
    Atividades & \multicolumn{6}{|c|}{Primeiro Semestre} \\
    \cline{2-7}                                                                             & Jan   & Fev   & Mar   & Abr   & Mai   & Jun   \\
    \hline Atualização e reescrita da dissertação conforme as orientações da banca          & X     & X     & X     & X     &       &       \\
    \hline Desenvolvimento do Jogo                                                          &       & X     & X     & X     & X     & X     \\
    \hline Comitê de Ética                                                                  &       &       & X     & X     & X     &      \\
    \hline Considerações Legais                                                             &       &       &       &       &       & X     \\
    \hline
    \end{tabular} 
    \\
    Fonte: Os autores (2020).
\end{table}

\captionsetup[table]{name=Quadro}
\begin{table}[!htb]
    \centering
    \renewcommand{\arraystretch}{1.5} %espaço entre as linhas
    \caption{\emph{Cronograma de Atividades para o segundo semestre de 2021}.}\label{tabelinha2}
    \vspace{0.2cm}
    \begin{tabular}{|p{8cm}|c|c|c|c|c|c|}
    \hline
    Atividades & \multicolumn{6}{|c|}{Segundo Semestre} \\
    \cline{2-7}                                                                             & Jul   & Ago   & Set   & Out   & Nov   & Dez   \\
    \hline Considerações Legais                                                             & X     &       &       &       &       &       \\
    \hline Segmentação                                                                      & X     &       &       &       &       &       \\
    \hline Pré-Teste                                                                        &       & X     &       &       &       &       \\
    \hline Teste                                                                            &       &       & X     & X     &       &       \\
    \hline Pós-Teste                                                                        &       &       &       &       & X     &       \\
    \hline Apreciação                                                                       &       &       &       &       & X     &       \\
    \hline Documentação                                                                     &       &       &       &       & X     & X     \\
    \hline Defesa                                                                           &       &       &       &       &       & X     \\
    \hline
    \end{tabular} 
    \\
    Fonte: Os autores (2020).
\end{table}


O Quadro \ref{tabelinha} apresenta o cronograma de atividades do presente trabalho para o primeiro semestre de 2021. No Quadro em questão são previstas aleterações no texto conforme as orientações da Banca e do Comitê de Ética. Além disso, é previso o cronograma de desenvolvimento do jogo assim como o início das considerações legais no mês de junho. 

O Quadro \ref{tabelinha2} apresenta o cronograma de atividades do presente trabalho para o primeiro semestre de 2021. O Quadro em questão consta todas as descritas na \autoref{sec:Avaliativos} e seus tempos previsto. 
