\chapter{Desenvolvimento}\label{ch:Desenvolvimento}

%A documentação do jogo é o Game Design Document (ou GDD para os íntimos). Alguns chamam também de Game Design Bible. É a mesma coisa


propósito

Orientações em Sexualidade

Plataforma

Publico Alvo

Estilo

Estética

Arte

Fonte

Escalabilidade/Flexibilidade

Audio

Musica

Leiaute de Níveis

UX

Gamificação

Ergonomia de Botões

Artefato

Licença

Termos de Serviço

Política de Privacidade

Criptografia

Banco de Dados


%Digital Natives. Our students today are all “native speakers” of the digital language of computers, video games and the Internet. %https://www.marcprensky.com/writing/Prensky%20-%20Digital%20Natives,%20Digital%20Immigrants%20-%20Part1.pdf %https://colegiongeracao.com.br/novageracao/2_intencoes/nativos.pdf


%Para que um game seja completo, e atenda a critérios de usabilidade, é essencial promover algum mecanismo que facilite o aprendizado do funcionamento do jogo. De acordo com Squire et al. (2005, p.41), mediadores são fundamentais nos primeiros dias%https://www.udesc.br/arquivos/cct/id_cpmenu/1024/diego_buchinger__1__15167055468902_1024.pdf


%A Metodologia Institucional “Aprender na Prática”, que prevê “a ação educativa na participação ativa e crítica do aluno em sua aquisição de conhecimentos práticos e teóricos” [UNICSUL, 2004] 