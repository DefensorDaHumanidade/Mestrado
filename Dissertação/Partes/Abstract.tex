% ---
% Abstract
% ---

% resumo em inglês
\begin{resumo}[Abstract]
 \begin{otherlanguage*}{english}
  Child sexual maltreatments are a worldwide problem. In this sense, some sexual prevention applications were created for providing a playful educational approach on the subject for children. This research introduces an serious game focused on teaching about prevention of sexual violence. The game developed follows pre-established precepts internationally for the education of minors about sexual issues. Among the teachings are concepts associated with the human body and bodily privacy, definition of rights and duties, strategies for personal safety and reporting, and care to be taken on the internet. The game is aimed at children from five to eight years old. It is validated through the Child Abuse Knowledge Questionnaire (CKAQ). The questionnaire is submitted to an experimental group and a control group in order to ascertain significant differences between the groups (95\% confidence level). If the results point to a difference between the groups, revealing increases in the preventive skills of the experimental group, it can be said that the game has the potential to assume a key role in preventing child sexual violence in Brazil.
  
   \textbf{Keywords}: Serious Game. Prevention. Children Sexual Abuse.
 \end{otherlanguage*}
\end{resumo}
