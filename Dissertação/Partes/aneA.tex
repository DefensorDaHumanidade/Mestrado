\chapter{Children’s Knowledge of Abuse Questionnaire Revised – III}\label{chap:CKAQ}

\noindent
\textbf{I.D. Number: \rule{5.4cm}{0.15mm} \quad Age: \rule{2.5cm}{0.15mm} \quad Boy \makebox[0pt][l]{$\square$}{}  \quad  or Girl \makebox[0pt][l]{$\square$}{}}

\vspace{1.0cm}

\noindent
\textbf{Please respond T for "True", F for "False", and DK for "Don't Know", to the following
questions:}



%\makebox[0pt][l]{$\square$}{\raisebox{0.1\height}{$\times$}} = Esse é o quadrado com o X dentro

\begin{enumerate}
	\item You always have to keep secrets.
	\item It's OK for someone you like to hug you.
	\item You can always tell who's a stranger - they look mean.
	\item Most kids like to get a kiss from their parents before they go to bed at night, so, for them, that would be a good touch.
	\item Sometimes it's OK to say no'' to a grown-up.
	\item It's OK to say ``no'' and move away if someone touches you in a way you don't like.
	\item Even if someone say that they know you, if you don't know them they're a stranger.
	\item Even hugs and tickles can turn into bad touches if they go on too long.
	\item If you fell off your bike and hurt your private parts, it would be OK for a doctor or nurse to look under your clothes.
	\item If someone touches you in a way you don't like, you should tell someone you trust.
	\item If your friend says he won't be your friend any more if you don't give him your last piece of candy, then you should give it to him.
	\item If someone touches you in a way you don't like, it's your own fault.
	\item If you don't like how someone is touching you, it's OK to say ``no''.
	\item Strangers look like ordinary people.
	\item If a grown-up tells you to do something you always have to do it.
	\item Some touches start out feeling good, then turn confusing.
	\item You can trust your feelings about whether a touch is good or bad.
	\item It's OK to have a hug from a grown-up you like.
	\item If a mean kid at school orders you to do something you had better do it.
	\item Even someone you like could touch you in a way that feels bad.
	\item A pat on the back from a teacher you like after you've done a good job at school is a good touch.
	\item You have to let grown-ups touch you whether you like it or not.
	\item If someone touches you in a way that does not feel good, you should keep on telling until someone believes you.
	\item Sometimes someone in your family might touch you in a way you don't like.
	\item Boys don't have to worry about someone touching their private parts.
	\item If you're walking down the street with your mother and she starts talking to a neighbour you have not met before, it's OK to talk with them too.
	\item If a friend's dad asks you to help him find their lost cat, you should go right away with him and help.
	\item If you won a contest for drawing the best picture in your school and a neighbour you liked gave you a quick hug to congratulate you, that would be a good touch.
	\item Most people are strangers and most strangers are nice.
	\item Someone you know, even a relative, might want to touch your private parts in a way that feels confusing.
	\item If your baby-sitter tells you to take off all your clothes but it's not time to get undressed for bed, you have to do it.
	\item If someone walks in while you are having a bath, and you feel uncomfortable, you should just keep quiet.
	\item If you get separated from your parents in a shopping mall, it's OK to ask a sales clerk or a security guard for help, even if they are strangers. 
\end{enumerate}


\chapter{Children’s Knowledge of Abuse Questionnaire Revised – III  (Traduzido)}\label{chap:traduzido}

\noindent
\textbf{Identificação da Escola: \rule{2.0cm}{0.15mm} \quad Idade: \rule{1.2cm}{0.15mm} \quad Masculino \makebox[0pt][l]{$\square$}{} \quad  ou Feminino \makebox[0pt][l]{$\square$}{}}

\vspace{0.5cm}

\noindent
\textbf{Por favor responda V para ``Verdadeiro'', F para ``Falso'' e N para ``Não Sei'', para as seguintes questões: }

\begin{enumerate}
	\item Você sempre deve guardar segredos.
	\item Está tudo bem para alguém que você gosta de te abraçar.
	\item Você sempre pode dizer quem é um estranho - eles parecem malvados.
	\item A maioria das crianças gosta de receber um beijo dos pais antes de ir para a cama à noite, então, para elas, seria um bom toque.
	\item Às vezes não há problema em dizer ``não'' a um adulto.
	\item É normal dizer ``não'' e afastar-se se alguém te tocar de uma forma que você não gosta.
	\item Mesmo que alguém diga que te conhece, se você não o conhece, é um estranho.
	\item Mesmo os abraços e cócegas podem se transformar em toques ruins se durarem muito.
	\item Se você cair de sua bicicleta e machucar suas partes íntimas, não há problema em um médico ou enfermeira examinar suas roupas.
	\item Se alguém te tocar de uma maneira que você não gosta, você deve contar a alguém de sua confiança.
	\item Se o seu amigo disser que não será mais seu amigo se você não lhe der o último doce, você deve dar a ele.
	\item Se alguém te toca de uma maneira que você não gosta, é sua própria culpa.
	\item Se você não gosta de como alguém está tocando em você, não há problema em dizer ``não''.
	\item Estranhos parecem pessoas comuns.
	\item Se um adulto lhe diz para fazer algo, você sempre tem que fazer.
	\item Alguns toques começam bem, depois tornam-se confusos.
	\item Você pode confiar em seus sentimentos sobre se um toque é bom ou ruim.
	\item Tudo bem receber um abraço de um adulto de quem você gosta.
	\item Se um garoto malvado na escola manda você fazer algo, é melhor você fazer.
	\item Até mesmo alguém de quem você gosta pode tocá-lo de uma forma que o faz sentir mal.
	\item Um tapinha nas costas de um professor de que você gosta depois de ter feito um bom trabalho na escola é um bom toque.
	\item Você tem que deixar os adultos tocarem em você, goste ou não.
	\item Se alguém tocar em você de uma forma que não lhe faça bem, continue contando até que alguém acredite em você.
	\item Às vezes, alguém da sua família pode tocar em você de uma maneira que você não gosta.
	\item Os meninos não precisam se preocupar com alguém tocando suas partes íntimas.
	\item Se você está andando na rua com sua mãe e ela começa a conversar com um vizinho que você ainda não conhece, não há problema em conversar com ele.
	\item Se o pai de um amigo lhe pedir para ajudá-lo a encontrar o gato perdido, você deve ir imediatamente com ele e ajudá-lo.
	\item Se você ganhou um concurso de desenho da melhor foto da sua escola e um vizinho de quem você gostou deu um rápido abraço em você para parabenizá-lo, seria um bom toque.
	\item A maioria das pessoas são estranhas e a maioria dos estranhos é legal.
	\item Alguém que você conhece, mesmo um parente, pode querer tocar suas partes íntimas de uma forma que pareça confusa.
	\item Se sua babá diz para você tirar toda a roupa, mas não é hora de se despir para dormir, você tem que fazer isso.
	\item Se alguém entrar enquanto você está tomando banho e você se sentir desconfortável, fique quieto.
	\item Se você se separar de seus pais em um shopping, não há problema em pedir ajuda a um balconista ou segurança, mesmo que sejam estranhos.
\end{enumerate}

\chapter{Aplicação do Questionário (CKAQ)}\label{chap:teste}

\noindent
\textbf{Instruções:}

O meu nome é \rule{3.0cm}{0.15mm} e preciso da tua/sua ajuda para perceber o que as crianças da tua/sua idade pensam sobre diferentes tipos de contactos/toques entre as pessoas. 

Sabes que existem, pelo menos 3 tipos de toques diferentes? As vezes sentes-te bem quando alguém te toca – esses são os \underline{bons} toques – como os abraços ou algumas palmadinhas nas costas. Alguns toques são \underline{maus} - como os beliscões e dentadas, porque te magoam ou te fazem sentir desconfortável. Até alguns beijos de pessoas de quem não gostas podem ser maus toques. Às vezes alguns toques são \underline{confusos} – esses acontecem quando te é difícil decidir se são bons ou maus toques. Por exemplo, alguém de quem gostas pode te dar um abraço, mas pode te apertar demais. És tu quem decide se um toque é bom ou mau, porque tu é que sabes como é que esse toque te faz sentir.

A outra palavra que eu quero ter a certeza que tu compreendes é underline{partes privadas}. Essas são as partes do teu corpo que o teu fato de banho tapa.

Vou colocar-te/lhe algumas questões sobre diferentes tipos de contactos/toques. Isto não é um teste para escola, não será/s avaliado por aquilo que responder/es ao que te/lhe vou perguntar. Responde/a de acordo com o que achar/es que está certo. Eu vou ler algumas perguntas e peço-te/lhe que responda/s “Sim” se achar/es que é verdade, “Não” se achar/es que é mentira e “Não sei” se não tiver/es a certeza.

\vspace{1.0cm}

\noindent
\textbf{NOTA:} Recomenda-se a administração verbal deste questionário para todas as crianças, especialmente se o CKAQ for usado para comparação entre crianças de diferentes idades em que as competências de leitura serão diferentes. Quando se lê os itens é importante acabarmos cada frase com ``\textbf{É verdade ou mentira?}''.

\vspace{1.0cm}

A aplicação deste questionário demora entre 10 a 15 minutos.


\begin{comment}
% ---
\chapter{Morbi ultrices rutrum lorem.}
% ---
\lipsum[30]

% ---
\chapter{Cras non urna sed feugiat cum sociis natoque penatibus et magnis dis
parturient montes nascetur ridiculus mus}
% ---

\lipsum[31]

% ---
\chapter{Fusce facilisis lacinia dui}
% ---

\lipsum[32]




\chapter{Savi}
\label{chap:A1}
%\section{Itens do questionário para avaliação do subcomponente motivação (ARCS) do modelo de \citeonline{savi2011avaliaccao}} 

Itens do questionário para avaliação do subcomponente motivação (ARCS) do modelo de \citeonline{savi2011avaliaccao}.

\begin{figure}[h]
	\centering
	%\caption{Modelo de avaliação de jogos educacionais}
	\includegraphics[width=1.0\textwidth]{img/SAVI-ARCS.png}
	\label{fig:anexo1}\\
	%Fonte: \cite{savi2011avaliaccao}.
\end{figure}

\newpage

%\section{Itens do questionário para avaliação do subcomponente experiência do usuário (UX) do modelo de \citeonline{savi2011avaliaccao}} 

Itens do questionário para avaliação do subcomponente experiência do usuário (UX) do modelo de \citeonline{savi2011avaliaccao}.\label{chap:A2}

\begin{figure}[h]
	\centering
	%\caption{Modelo de avaliação de jogos educacionais}
	\includegraphics[width=1.0\textwidth]{img/SAVI-UX.png}
	\label{fig:anexo2}\\
	%Fonte: \cite{savi2011avaliaccao}.
\end{figure}

\newpage

%\section{Itens do questionário para avaliação do subcomponente aprendizagem do modelo de \citeonline{savi2011avaliaccao}} 

Itens do questionário para avaliação do subcomponente aprendizagem do modelo de \citeonline{savi2011avaliaccao}.\label{chap:A3}

\begin{figure}[h]
	\centering
	%\caption{Modelo de avaliação de jogos educacionais}
	\includegraphics[width=1.0\textwidth]{img/Savi Aprendizagem.png}
	\label{fig:anexo3}\\
	%Fonte: \cite{savi2011avaliaccao}.
\end{figure}

\newpage

%\section{Mapeamento dos itens do questionário ao modelo teórico do modelo de \citeonline{savi2011avaliaccao}} 

Mapeamento dos itens do questionário ao modelo teórico do modelo de \citeonline{savi2011avaliaccao}.\label{chap:A4}

\begin{figure}[h]
	\centering
	%\caption{Modelo de avaliação de jogos educacionais}
	\includegraphics[width=1.0\textwidth]{img/Savi-mapeamento.jpg}
	\label{fig:anexo4}\\
	%Fonte: \cite{savi2011avaliaccao}.
\end{figure}

\newpage

\chapter{Takatalo}
\label{chap:A5}
%\section{Componentes da Experiência do Usuário em jogos do modelo de \citeonline{takatalo2010presence}} 
Componentes da Experiência do Usuário em jogos do modelo de \citeonline{takatalo2010presence}.
Modelos de experiência do usuário em jogos (na língua original da publicão). 

\begin{figure}[h]
	\centering
	%\caption{Modelo de avaliação de jogos educacionais}
	\includegraphics[width=1.0\textwidth]{img/savi-takatalo.jpg}
	\label{fig:anexo5}\\
	%Fonte: \cite{savi2011avaliaccao}.
\end{figure}

\newpage

\chapter{Poels, Kort e Ijsselsteijn}
\label{chap:A6}
%\section{Componentes da Experiência do Usuário em jogos do modelo de \citeonline{poels2007always}} 
Componentes da Experiência do Usuário em jogos do modelo de \citeonline{poels2007always} (na língua original da publicão).

\begin{figure}[htb]
	\centering
	%\caption{Modelo de avaliação de jogos educacionais}
	\includegraphics[width=0.9\textwidth]{img/Savi-Kort.jpg}
	\label{fig:anexo6}\\
	%Fonte: \cite{savi2011avaliaccao}.
\end{figure}

\newpage

\chapter{Gamez}
\label{chap:A7}
%\section{Componentes da Experiência do Usuário em jogos do modelo de \citeonline{calvillo2009core}} 
Componentes da Experiência do Usuário em jogos do modelo de \citeonline{calvillo2009core} (na língua original da publicão).

\begin{figure}[htb]
	\centering
	%\caption{Modelo de avaliação de jogos educacionais}
	\includegraphics[width=1.0\textwidth]{img/Savi-Gamez.jpg}
	\label{fig:anexo7}\\
	%Fonte: \cite{savi2011avaliaccao}.
\end{figure}

\newpage

\chapter{Sweetser e Wyeth}
\label{chap:A8}
%\section{Componentes da Experiência do Usuário em jogos do modelo de \citeonline{sweetser2005gameflow}} 
Componentes da Experiência do Usuário em jogos do modelo de \citeonline{sweetser2005gameflow} (na língua original da publicão).

\begin{figure}[htb]
	\centering
	%\caption{Modelo de avaliação de jogos educacionais}
	\includegraphics[width=1.0\textwidth]{img/Savi-Witef.jpg}
	\label{fig:anexo8}\\
	%Fonte: \cite{savi2011avaliaccao}.
\end{figure}

\newpage

\begin{figure}[htb]
	\centering
	%\caption{Modelo de avaliação de jogos educacionais}
	\includegraphics[width=1.0\textwidth]{img/Savi-Witef2.png}
	\label{fig:anexo9}\\
	%Fonte: \cite{savi2011avaliaccao}.
\end{figure}
\end{comment}

