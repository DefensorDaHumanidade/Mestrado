\chapter{Considerações Fianis}\label{ch:Conclusao}


Apesar de alguns autores, os quais compartilho da ideia, criticarem que os jogos eletrˆonicos causam aliena¸c˜ao e levam ao vicio, tais jogos em raz˜ao das possibilidades de intera¸c˜ao e recep¸c˜ao, vˆem sendo aplicados como instrumentos fact´ıveis de melhorar a aprendizagem. %https://files.cercomp.ufg.br/weby/up/498/o/Cuba2009.pdf
Bittencourt e Giraffa (2003), “a sociedade atual ainda est´a muito presa aos valores e processos da era industrial, quando se defendia que trabalho e divers˜ao eram campos distintos” .

Conograma

Passar pelo comite de etica

Criança se sentir culpada...

....

Using the Internet for abuse prevention provides several benefits. First, as many children as desired can use an existing online prevention program without causing extra costs. This could help schools with lower budgets or in rural locations, that might have problems to offer face-to-face prevention programs. However, this only can be true if there is no commercial interest behind the program. A second benefit is the fact that online programs can easily be translated for a) children with immigration backgrounds, or b) an implementation in various countries. Third, the use of a web-based program allows teachers a maximum of flexibility when including the program in their curricula. Because of the individual conduction, children can follow the program in their specific learning speed. Also, it can be perfectly included in open learning environments as well as in conventional classroom settings. But not only schools can benefit from the existence of online prevention programs. Parents also address the topic of sexual abuse with their children (Walsh \& Brandon, 2012) and might find programs like this one helpful as well.\cite{muller2014child}

No entanto, nossa reivindicação não é substituir os programas de prevenção tradicionais por treinamento baseado na web. A interação social e a possibilidade de discutir e treinar comportamentos, por exemplo, por meio de dramatizações, são facetas importantes dos programas de prevenção (Davis \& Gidycz, 2000). Mas os resultados deste estudo mostram que a prevenção online pode ser uma alternativa eficaz quando não há um programa presencial disponível, ou pode muito bem servir como uma repetição que pode ser implementada algum tempo depois de uma prevenção presencial programa. Claro, algumas questões permanecem sem resposta por este estudo. Em primeiro lugar, seria muito interessante comparar um programa de prevenção baseado na web não apenas a um grupo de controle de lista de espera, mas ter um treinamento presencial com o mesmo conteúdo e duração de uma condição de controle. Isso ajudaria a avaliar se os efeitos da prevenção baseada na web são tão bons quanto os dos programas tradicionais. Em segundo lugar, por razões éticas, não é possível avaliar como as crianças reagem às tentativas de abuso sexual na realidade. Embora Fryer, Kraizer e Miyoshi (1987) pudessem prever se as crianças iriam com um estranho atrás de um programa de prevenção de perigo para um estranho por seu conhecimento, o comportamento real é difícil de medir quando se trata de abuso sexual, especialmente por pessoas conhecidas pela criança . No entanto, no contexto da segurança na Internet, pode haver maneiras de testar a disposição das crianças de dar informações privadas e, portanto, avaliar os efeitos dos programas de prevenção no comportamento real. É claro que os projetos de estudo teriam que ser considerados com muito cuidado em todos os casos, mas essa poderia ser uma questão interessante que precisa ser respondida por pesquisas futuras. \cite{muller2014child}


