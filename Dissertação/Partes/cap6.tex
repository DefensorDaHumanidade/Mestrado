\chapter{Considerações Fianis}\label{ch:Conclusao}

%Usar a Internet para prevenção de abusos oferece vários benefícios. Em primeiro lugar, quantas crianças desejar podem usar um programa de prevenção online existente sem causar custos extras. Isso poderia ajudar as escolas com orçamentos mais baixos ou em locais rurais, que podem ter problemas para oferecer programas de prevenção face a face. No entanto, isso só pode ser verdade se não houver interesse comercial por trás do programa. Um segundo benefício é o fato de que os programas online podem ser facilmente traduzidos para a) crianças com histórico de imigração, ou b) uma implementação em vários países. Terceiro, o uso de um programa baseado na web permite aos professores um máximo de flexibilidade ao incluir o programa em seus currículos. Por conta da condução individual, as crianças podem acompanhar o programa em sua velocidade de aprendizagem específica. Além disso, pode ser perfeitamente incluído em ambientes de aprendizagem abertos, bem como em ambientes de sala de aula convencionais. Mas não apenas as escolas podem se beneficiar da existência de programas de prevenção online. Os pais também abordam o tópico de abuso sexual com seus filhos (Walsh \ & Brandon, 2012) e podem achar programas como este úteis também.

O abuso sexual infantil é um grave problema de predominância global que assola milhares de crianças todos os anos. Os danos da violência sexual infantil são largamente conhecidos e documentados na literatura médica da área, os quais podem acompanhar a criança violentada durante a vida inteira. Em resposta a esse problema, inúmeras estratégias surgiram. A capacitação de crianças é uma estratégia promissora que se destaca entre as demais estratégias. O aperfeiçoamento da segurança pessoal por meio de programas de capacitação é uma atitude capaz de evitar a ocorrência de episódios de abuso. Por meio dos programas de capacitação para crianças o problema da violência sexual é cortado pela raiz, pois os agressores sexuais evitam abordar crianças com maiores probabilidades de recusar e relatar suas abordagens abusivas para as devidas autoridades. 

Em vias de mitigar o problema da violência sexual infantil no Brasil a presente pesquisa almeja desenvolver um programa de capacitação para crianças de cinco a oito anos de idade. O programa em questão possui seus conceitos pedagógicos baseados em orientações técnicas internacionais de educação em sexualidade. A dinâmica do programa assume o caráter de um jogo sério. Uma abordagem baseada em jogos fornece um meio de aprendizagem promovendo uma abordagem educacional divertida e envolvente para a prevenção da violência infantil. A utilização de jogos, permite que alunos possam aprender através da vivência de situações simuladas, sem ter que passar por elas efetivamente.

Em relação aos programas tradicionais para a prevenção da violência sexual infantil, o programa proposto pela presente pesquisa se utiliza de um jogo educacional digital, o qual permite que os menores possam se manifestar em sua plenitude, sem se sentirem intimidados ou acanhados. 
%referencia muller2014child
Além disso, uma estratégia digital ainda permite que os custo de expansão do programa sejam mais reduzidos em comparação a estratégias presenciais. 


%Em virtude da sensibilidade do tema tratado, o jogo foi analisado pela presente pesquisa para assegurar que os conteúdos ministrados no jogo estariam devidamente adequados para seu público alvo. Sua validação foi executada com dois grupos de crianças na faixa etária do jogo. Tanto o grupo controle, quanto o grupo experimental foram foi submetidos ao questionário CKAQ na etapa de pré-teste e pós-teste desta pesquisa. Sob um grau de confiança de 95\% (Teste-t), observou-se diferença significativa entre os grupos. 

%Os resultados revelaram a importância no estudo da arte e no cumprimento das bases da literatura, para o desenvolvimento de um jogo adequado ao público infantil.

A atual pesquisa ainda encontra-se em processo de desenvolvimento. Em virtude de sua temática sensível e da vulnerabilidade do público alvo, se faz necessário que o atual trabalho passe pelos devidos protocolos do Comitê de Ética. Inclusive, em virtude de questões éticas não se faz possível que essa pesquisa seja capaz de mensurar o comportamento infantil a tentativas simuladas de abuso. Por tal razão, os dados a serem obtidos sobre a aprendizagem das crianças no programa não podem ser inferidos para suas atitudes comportamentais em situações de abuso. 

\pagebreak

Outra questão a ser levada em consideração se relaciona com a responsabilidade imposta as crianças pelos programas de capacitação. Há a preocupação de alguns pesquisadores na área que programas do gênero podem trazer um sentimento de culpa as crianças envolvidas a episódio de abuso, piorando assim o quadro clínico dos menores nestes casos. 




%Claro, algumas questões permanecem sem resposta por este estudo. Em primeiro lugar, seria muito interessante comparar um programa de prevenção baseado na web não apenas a um grupo de controle de lista de espera, mas ter um treinamento presencial com o mesmo conteúdo e duração de uma condição de controle. Isso ajudaria a avaliar se os efeitos da prevenção baseada na web são tão bons quanto os dos programas tradicionais. Em segundo lugar, por razões éticas, não é possível avaliar como as crianças reagem às tentativas de abuso sexual na realidade.


Almeja-se que os ensinamentos de prevenção a violência sexual sejam incluídos na Base Nacional Comum Curricular (e não apenas conhecimentos de reprodução e demais afins). Deste modo, surge a chance para a inclusão do jogo desenvolvido em salas do ensino fundamental, com o jogo agindo como um agregador e não como um substituto das aulas tradicionais, trazendo assim mais engajamento e ludicidade as aulas de prevenção à violência sexual infantil.

    

O jogo a ser desenvolvido pelo presente trabalho é uma continuação da dissertação do professor Tiago Francisco Andrade Diocesano. %, o qual batizou o jogo de \textit{Infância Segura}. 
O jogo desenvolvido nesta pesquisa é de propriedade da Universidade do Estado de Santa Catarina. Entretanto, o jogo é de código aberto e de licença livre, permitindo sua adaptação e expansão para outros idiomas. Os próximos passos da presente pesquisa são apresentados na \autoref{tabelinha}.


\begin{table}[!htb]
    \centering
    \renewcommand{\arraystretch}{1.5} %espaço entre as linhas
    \begin{tabular}{|p{9cm}|c|c|c|c|c|c|}
    \hline
    Atividades & \multicolumn{6}{|c|}{Primeiro Semestre} \\
    \cline{2-7}                                                                             & Jan   & Fev   & Mar   & Abr   & Mai   & Jun   \\
    \hline Atualização e reescrita da dissertação conforme as orientações da banca          & X     &       &       &       &       &       \\
    \hline Desenvolvimento do Jogo                                                          &       & X     & X     & X     &       &       \\
    \hline Validação do Jogo                                                                &       &       &       &       & X     &       \\
    \hline Documentação dos resultados e achados na pesquisa                                &       &       &       &       &       & X     \\
    \hline
    \end{tabular} \caption{\emph{Cronograma de Atividades para o primeiro semestre de 2021}}\label{tabelinha}
\end{table}


A \autoref{tabelinha} apresenta o cronograma de atividades do presente trabalho para o primeiro semestre de 2021. Para o mês de janeiro é previsto alterações na parte textual da dissertação, quanto também a submissão da presente pesquisa para o Comitê de Ética. De Fevereiro até Abril é esperado o início e conclusão do desenvolvimento do jogo. Para Março é prevista a etapa de validação com crianças do jogo. E por fim no mês de junho se espera a documentação dos principais achados e a defesa da presente pesquisa. 


%Apesar de alguns autores, os quais compartilho da ideia, criticarem que os jogos eletrˆonicos causam aliena¸c˜ao e levam ao vicio, tais jogos em raz˜ao das possibilidades de intera¸c˜ao e recep¸c˜ao, vˆem sendo aplicados como instrumentos fact´ıveis de melhorar a aprendizagem. %https://files.cercomp.ufg.br/weby/up/498/o/Cuba2009.pdf
%Bittencourt e Giraffa (2003), “a sociedade atual ainda est´a muito presa aos valores e processos da era industrial, quando se defendia que trabalho e divers˜ao eram campos distintos” .



%Using the Internet for abuse prevention provides several benefits. First, as many children as desired can use an existing online prevention program without causing extra costs. This could help schools with lower budgets or in rural locations, that might have problems to offer face-to-face prevention programs. However, this only can be true if there is no commercial interest behind the program. A second benefit is the fact that online programs can easily be translated for a) children with immigration backgrounds, or b) an implementation in various countries. Third, the use of a web-based program allows teachers a maximum of flexibility when including the program in their curricula. Because of the individual conduction, children can follow the program in their specific learning speed. Also, it can be perfectly included in open learning environments as well as in conventional classroom settings. But not only schools can benefit from the existence of online prevention programs. Parents also address the topic of sexual abuse with their children (Walsh \& Brandon, 2012) and might find programs like this one helpful as well.\cite{muller2014child}

%No entanto, nossa reivindicação não é substituir os programas de prevenção tradicionais por treinamento baseado na web. A interação social e a possibilidade de discutir e treinar comportamentos, por exemplo, por meio de dramatizações, são facetas importantes dos programas de prevenção (Davis \& Gidycz, 2000). Mas os resultados deste estudo mostram que a prevenção online pode ser uma alternativa eficaz quando não há um programa presencial disponível, ou pode muito bem servir como uma repetição que pode ser implementada algum tempo depois de uma prevenção presencial programa. Claro, algumas questões permanecem sem resposta por este estudo. Em primeiro lugar, seria muito interessante comparar um programa de prevenção baseado na web não apenas a um grupo de controle de lista de espera, mas ter um treinamento presencial com o mesmo conteúdo e duração de uma condição de controle. Isso ajudaria a avaliar se os efeitos da prevenção baseada na web são tão bons quanto os dos programas tradicionais. Em segundo lugar, por razões éticas, não é possível avaliar como as crianças reagem às tentativas de abuso sexual na realidade. Embora Fryer, Kraizer e Miyoshi (1987) pudessem prever se as crianças iriam com um estranho atrás de um programa de prevenção de perigo para um estranho por seu conhecimento, o comportamento real é difícil de medir quando se trata de abuso sexual, especialmente por pessoas conhecidas pela criança . No entanto, no contexto da segurança na Internet, pode haver maneiras de testar a disposição das crianças de dar informações privadas e, portanto, avaliar os efeitos dos programas de prevenção no comportamento real. É claro que os projetos de estudo teriam que ser considerados com muito cuidado em todos os casos, mas essa poderia ser uma questão interessante que precisa ser respondida por pesquisas futuras. \cite{muller2014child}


