%!TEX root = ../Principal.tex
%Capa do Trabalho
\imprimircapa

%Folha de Rosto
%* indica que tem ficha catalográfica
\imprimirfolhaderosto*

% ---
% Caso a Biblioteca da UDESC forneça, utilize o comando
% ---
% \begin{fichacatalografica}
%     \includepdf{fig_ficha_catalografica.pdf}
% \end{fichacatalografica}

% ---
% Geração da Ficha Catalográfica Via LaTeX
% ---
\begin{fichacatalografica}
	\vspace*{\fill}					% Posição vertical
	\begin{center}					% Minipage Centralizado
	\begin{minipage}[c]{12.5cm}		% Largura
	
	\imprimirautor
	
	\hspace{0.5cm} \imprimirtitulo  / \imprimirautor. --
	\imprimirlocal, \imprimirdata-
	
	\hspace{0.5cm} %\pageref{LastPage} p. : il. (algumas color.) ; 30 cm.\\
	
	\hspace{0.5cm} \imprimirorientadorRotulo~\imprimirorientador\\

	\hspace{0.5cm} \imprimircoorientadorRotulo~\imprimircoorientador\\
	
	\hspace{0.5cm}
	\parbox[t]{\textwidth}{\imprimirtipotrabalho~--~\imprimirinstituicao,
	\imprimirdata.}\\
	
	\hspace{0.5cm}
		1. Tópico 01.
		2. Tópico 02.
		I. Prof. Dr. xxxxx.
		II. Universidade do Estado de Santa Catarina.
		III. Centro de Ciências Tecnológicas.
		IV. identificação xxxx\\ 			
	
	\hspace{8.75cm} CDU 02:121:005.7\\
	
	\end{minipage}
	\end{center}
\end{fichacatalografica}

% ---
% Folha de Aprovação
% ---
% Exemplo de folha de aprovação antes da Banca. Após isso, incluia o pdf digitalizado com as assinaturas%
% \includepdf{folhadeaprovacao_final.pdf}
\begin{folhadeaprovacao}

	\begin{center}
		{\ABNTEXchapterfont\bfseries\imprimirautor}
		\vspace{6em}

			\ABNTEXchapterfont\bfseries\imprimirtitulo
		
	\end{center}
		\vspace{1em}
		{\justify
		Esta dissertação foi julgada adequada para a obtenção do título de
    	{\ABNTEXchapterfont\bfseries Mestre em Computação Aplicada}   
   		área de de concentração em ``Sistemas de Computação'',
   		 e aprovada em sua forma final pelo Curso de Mestrado em Computação Aplicada do Centro
   		 de Ciências Tecnológicas da Universidade ddo Estado de Santa Catarina.}
	
	\vspace{3em} 
	\noindent
	{\bfseries Banca Examinadora:}
	\assinatura{\textbf{\imprimirorientador} \\ Orientadora}
	\assinatura{\textbf{\imprimircoorientador} \\ Coorientador}  
	\assinatura{\textbf{Luciano Heitor Gallegos Marin} \\ Membro da Banca Examinadora}
    \assinatura{\textbf{Vera Márcia Marques Santos} \\ Membro da Banca Examinadora}
    %\assinatura{\textbf{Professor} \\ Convidado 3}

    \vspace*{\fill}
    \begin{center}
    	\imprimirlocal,\,\imprimirfulldata
    \end{center}
\end{folhadeaprovacao}

% ---
% Dedicatória
% ---
\begin{dedicatoria}				
Dedico este trabalho aos meus familiares, amigos, colegas e professores que me acompanharam e me deram forças nessa magnífica trajetória.  
\end{dedicatoria}

% ---
% Agradecimentos
% ---
\begin{agradecimentos}
%Gostaria de agradecer...

O presente trabalho foi realizado com apoio da Coordenação de Aperfeiçoamento de Pessoal de Nível Superior - Brasil (CAPES) - Código de Financiamento 001.

%Aqui devem ser colocadas os agradecimentos às pessoas que de alguma forma contribuíram para a realização do trabalho.
\end{agradecimentos}

% ---
% Epígrafe
% ---
\begin{epigrafe}	
%``Independentemente das circunstâncias, devemos ser sempre humildes, recatados e despidos de orgulho.''
``O que se faz agora com as crianças é o que elas farão depois com a sociedade.''
\\
\par
Karl Mannheim
%Dalai Lama 
\end{epigrafe}

% ---
% RESUMOS
% ---

% Português
\begin{resumo}
 %O resumo deve ressaltar o
 %objetivo, o método, os resultados e as conclusões do documento. A ordem e a extensão
 %destes itens dependem do tipo de resumo (informativo ou indicativo) e do
 %tratamento que cada item recebe no documento original. O resumo deve ser
 %precedido da referência do documento, com exceção do resumo inserido no
 %próprio documento. (\ldots) As palavras-chave devem figurar logo abaixo do
 %resumo, antecedidas da expressão Palavras-chave:, separadas entre si por
 %ponto e finalizadas também por ponto.

A violência sexual infantil é um problema de saúde pública que sequela tanto a vítima quanto a sociedade. Nesse sentido, jogos para prevenção da violência sexual infantil surgem com o objetivo de proporcionarem uma abordagem educacional lúdica acerca do assunto para as crianças. A presente pesquisa introduz um jogo educacional focado na ensino-aprendizagem da prevenção da violência sexual. O jogo desenvolvido segue preceitos pré-estabelecidos internacionalmente para a educação de menores acerca da temática sexual. Dentre os ensinamentos estão conceitos associados ao corpo humano e privacidade corporal, definição dos direitos e deveres, estratégias de segurança pessoal e denúncia, e cuidados a serem tomados na \textit{internet}. O jogo é voltado para crianças de cinco a oito anos de idade. Sua validação se dá por meio do Questionário Sobre Conhecimentos de Abuso Infantil (CKAQ). O questionário é submetido a um grupo experimental e a um grupo controle com o intuito de averiguar diferenças significativas entre os grupos (grau de confiança de 95\%). Caso os resultados apontem para uma diferença entre os grupos, revelando incrementos nas habilidades preventivas do grupo experimental, pode-se dizer que o jogo possui potencial para assumir um papel chave na prevenção da violência sexual infantil no Brasil.

 \vspace{\onelineskip}
    
 \noindent
 \textbf{Palavras-chave}: Tecnologia Educacional, violência sexual infantil e prevenção.
\end{resumo}

% Inglês
\begin{resumo}[Abstract]
 \begin{otherlanguage*}{english}
	%Resumo em inglês
	Child sexual maltreatments are a worldwide problem. In this sense, some sexual prevention applications were created for providing a playful educational approach on the subject for children. This research introduces an serious game focused on teaching about prevention of sexual violence. The game developed follows pre-established precepts internationally for the education of minors about sexual issues. Among the teachings are concepts associated with the human body and bodily privacy, definition of rights and duties, strategies for personal safety and reporting, and care to be taken on the internet. The game is aimed at children from five to eight years old. It is validated through the Child Abuse Knowledge Questionnaire (CKAQ). The questionnaire is submitted to an experimental group and a control group in order to ascertain significant differences between the groups (95\% confidence level). If the results point to a difference between the groups, revealing increases in the preventive skills of the experimental group, it can be said that the game has the potential to assume a key role in preventing child sexual violence in Brazil.
   \vspace{\onelineskip}
 
   \noindent 
   \textbf{Key-words}: Serious game, prevention and child sexual violence.
 \end{otherlanguage*}
\end{resumo}

% ---
% Lista de Figuras
% ---
\pdfbookmark[0]{\listfigurename}{lof}
\listoffigures*
\cleardoublepage
% ---

% ---
% Lista de Tabelas
% ---
\pdfbookmark[0]{\listtablename}{lot}
\listoftables*
\cleardoublepage