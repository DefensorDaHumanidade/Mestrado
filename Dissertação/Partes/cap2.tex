\chapter{Fundamentação}\label{ch:Fundamentacao}

A fundamentação teórica é um elemento fundamental para a compreensão dos fundamentos teóricos que dão base aos conceitos que permeiam o objeto de estudo em uma determinada pesquisa. O capítulo de fundamentação surge então para munir o pesquisador dos conhecimentos necessários para o andamento e a conclusão de sua pesquisa. Ao mesmo tempo, a fundamentação auxilia outros pesquisadores na idenficação das bases do conhecimento teórico que guiaram determinando estudo. 

O corrente estudo tem como cerne a validação de um programa educacional para a prevenção da violência sexual infantil baseado na dinâmicas de jogos sérios. Deste modo, a seção \ref{sec:JogosSerios} fundamenta o conceito de Jogos Sérios. A seção \ref{sec:Engenharia} trás alguns conceitos sobre o desenvolvimento de jogos.

\section{Jogos Sérios}\label{sec:JogosSerios}

\section{Engenharia de Sotfware}\label{sec:Engenharia}

\section{Learning Analytics}\label{sec:LA}

