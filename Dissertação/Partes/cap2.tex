\chapter{Fundamentação}\label{ch:Fundamentacao}

A fundamentação teórica é um elemento fundamental para a compreensão dos fundamentos teóricos que dão base aos conceitos que permeiam o objeto de estudo em uma determinada pesquisa. O capítulo de fundamentação surge então para munir o pesquisador dos conhecimentos necessários para o andamento e a conclusão de sua pesquisa. Ao mesmo tempo, a fundamentação auxilia outros pesquisadores na idenficação das bases do conhecimento teórico que guiaram determinando estudo. 

O corrente estudo tem como cerne a validação de um programa educacional para a prevenção da violência sexual infantil baseado na dinâmicas de jogos sérios. Deste modo, a seção \ref{sec:JogosSerios} fundamenta o conceito de Jogos Sérios. A seção \ref{sec:Engenharia} trás alguns conceitos sobre o desenvolvimento de jogos.

\section{Jogos Sérios}\label{sec:JogosSerios}


o Tavares (2007a) o entretenimento ´e utilizado com uma finalidade de passar conhecimentos, informa¸c˜oes, valores e
atitudes.%https://files.cercomp.ufg.br/weby/up/498/o/Cuba2009.pdf
Nesse enfoque e ainda com o que diz respeito a educa¸c˜ao, Tarouco et al. (2004) destaca que “os games podem se tornar ferramentas instrucionais eficientes, pois eles divertem e motivam, facilitando o aprendizado, pois aumenta a capacidade de reten¸c˜ao do que foi ensinado”.


Existem vários conceitos que definem jogos em um contexto educacional: jogos educativos, jogos baseados em aprendizagem, jogos de simulação.... [mostrar um complilado dos driagramas de Ven].

Aparentemente, não há um consenso cientifício no no que diz respeito a taxonomia e seus escopos. 



\section{Engenharia de Sotfware}\label{sec:Engenharia}

\section{Learning Analytics}\label{sec:LA}

