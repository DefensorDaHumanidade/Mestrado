\chapter{Fundamentação}\label{ch:Fundamentacao}

A fundamentação teórica é um elemento fundamental para a compreensão dos fundamentos teóricos que dão base aos conceitos que permeiam o objeto de estudo em uma determinada pesquisa. O capítulo de fundamentação surge então para munir o pesquisador dos conhecimentos necessários para o andamento e a conclusão de sua pesquisa. Ao mesmo tempo, a fundamentação auxilia outros pesquisadores na idenficação das bases do conhecimento teórico que guiaram determinando estudo. 

O corrente estudo tem como cerne a validação de um programa educacional para a prevenção da violência sexual infantil baseado na dinâmicas de jogos sérios. Deste modo, a seção \ref{sec:JogosSerios} fundamenta o conceito de Jogos Sérios. A seção \ref{sec:Engenharia} trás alguns conceitos sobre o desenvolvimento de jogos. % A seção 2.3 discutei os sistemas de aplicação de programas educacionais e suas formas de validação. 

\section{Jogos Sérios}\label{sec:JogosSerios}

O termo \underline{Jo}g\underline{o Sério} (em inglês: \textit{Serious Game}) surge pela primeira vez na história em 1970 \cite{clark1970serious}. Desde então, o termo passou por inúmeras revisões até alcançar uma definição geral, a qual classifica como Jogo Sério: todos os jogos projetados para uma finalidade principal que não a pura diversão \cite{michael2005serious, de2015aprendizagem, laamarti2014overview}.

A definição geral de Jogos Sérios permite identificar que os jogos classificados como sérios antecedem a própria origem do termo. Isso pois, a história conta que desde muito antes da década de 70, alguns jogos já eram utilizados para outros propósitos além do entreterimento \cite{wilkinson2016brief}. Salienta-se no entanto que o termo não encontra-se verdadeiramente consolidado na literatura científica da área, existindo inclusive várias definições e termos correlatos \cite{pourabdollahian2012serious}.

Para definir melhor o termo \underline{Jo}g\underline{o Sério} que fundamentará e guiará o andamentamento do presente estudo, buscou-se separar o conceito e interpretar individualmente as palavras que o compõem. A Figura x, ilustra de forma resumida os conceitos que fundamentam a definição de jogo sério do atual trabalho.  

%Jogo: “é uma atividade mais estruturada e estabelecida por um princípio de regras mais explícitas” que integra tanto o objeto quanto a brincadeira (Kishimoto, 1994) ou “ação de jogar” (Bertoldo & others, 2000) %https://repositorio-aberto.up.pt/bitstream/10216/110820/2/253022.pdf

%Um game é uma atividade lúdica composta por uma série de ações e decisões, limitado por regras e pelo universo do game, que resultam em uma condição final. As regras e o universo do game são apresentados por meios eletrônicos e controlados por um programa digital (...) [e] existem para proporcionar uma estrutura e um contexto para as ações de um jogador. As regras também existem para criar situações interessantes com o objetivo de desafiar e se contrapor ao jogador. Jogos Sérios são aplicações que mesclam aspectos sérios como o ensino, a aprendizagem, a comunicação e a informação, com o lúdico e interativo fornecido pelos games, sendo o principal objetivo outro além do puro entretenimento. Criar um Jogo Sério é conseguir fundir de forma atraente e interessante informações para ensino, com interações dinâmicas e lúdicas (BOYLE; CONNOLLY; HAINEY, 2011, p. 71).%https://www.udesc.br/arquivos/cct/id_cpmenu/1024/diego_buchinger__1__15167055468902_1024.pdf

%Esta combinação tem como propósito tornar o conteúdoprático, útil (sério) e jogável, o que é alcançado por meio do desenvolvimento decenários que são ao mesmo tempo práticos e agradáveis.%https://webcache.googleusercontent.com/search?q=cache:fWC_TzjZ76QJ:https://www.br-ie.org/pub/index.php/pie/article/download/6595/4506+&cd=1&hl=pt-BR&ct=clnk&gl=br

%computer application, for which the original intention is to combine with consistency, both serious (Serious) aspects such as non-exhaustive and non-exclusive, teaching, learning, communication, or the information, with playful springs from the video game (Game). %http://hayka-kultura.org/images/Proceedings%20SGS%20Wkshp%202011%20ind%2004.pdf#page=11



%[GRAFICOOOOOO]http://downloads.hindawi.com/journals/ijcgt/2014/358152.pdf


Os Jogos Sérios são capazes de alcançar e enriquecer setores inteiros 


Para este trabalho, adotaram-se as visões sobre Jogos Sérios definidas nessa seção. 


Assim, os jogos digitais são recursos didáticos com características que podem trazer benefícios ao processo de ensino-aprendizagem, com efeito motivador, facilitador do aprendizado, desenvolvimento de habilidades cognitivas e aprendizado por descobertas, como descritos por Savi e Ulbricht (2008).


Entretanto, até o ano de 2020, o termo não se encontra consolidado na literatura científica;


Embora não hajam um concenso científico na área a respeito da terminologia e de sua definição desde a década de 70



pode-se dizer que uma definição geral aceita é que 



não está consolidado na literatura científica da área. Isso torna o processo de classificação de jogos sérios uma tarefa 


O termo \underline{Jo}g\underline{o Sério} (em inglês: \textit{Serious Game}) estabelece uma classe de jogos projetados para uma finalidade principal que não a pura diversão \cite{michael2005serious}. Sendo esta, a definição geral mais aceita, e identificada nas obras consultadas. 



ainda não está bem consolidado na literatura científica; existem várias definições e termos correlatos (POURABDOLLAHIAN, TAISCH, KERGA, 2012, p. 257).


%Por exemplo, explorar a aplicação de jogos para fins diferentes do entretenimento tem uma precedência histórica na aplicação do jogo - especialmente em contextos educacionais.


A palavra jogo define 



Em outras palavras, são classificados desta forma, os jogos com propósitos sério, com o intuito de capacitar, educar ou aprimorar habilidades dos seus jogadores. 



ue usa la diversión como modo de formación gubernamental o corporativo, con objetivos en el ámbito de la educación, sanidad, política pública y comunicación estratégic


Ou seja, jogos sérios são jogos com um propósito sério de ensino, 


o Tavares (2007a) o entretenimento ´e utilizado com uma finalidade de passar conhecimentos, informa¸c˜oes, valores e
atitudes.%https://files.cercomp.ufg.br/weby/up/498/o/Cuba2009.pdf
Nesse enfoque e ainda com o que diz respeito a educa¸c˜ao, Tarouco et al. (2004) destaca que “os games podem se tornar ferramentas instrucionais eficientes, pois eles divertem e motivam, facilitando o aprendizado, pois aumenta a capacidade de reten¸c˜ao do que foi ensinado”.


Existem vários conceitos que definem jogos em um contexto educacional: jogos educativos, jogos baseados em aprendizagem, jogos de simulação.... [mostrar um complilado dos driagramas de Ven].

Aparentemente, não há um consenso cientifício no no que diz respeito a taxonomia e seus escopos. 

%A classificação dos videogames está longe de ser uma nova ideia. Muitos sistemas de "classificações empíricas" já existem, e são realmente usados pela indústriade videogames, críticos e gamers. No entanto, mesmo que esses muitos sistemas sejam, sem dúvida, uma parte importante da cultura comum do videogame,eles infelizmente não são adequados para a classificação unificada de todos os videogames lançados. De fato, esses sistemas "empíricos" acompanham de perto a evolução dos videogames: novas categorias aparecem, outras são removidas, e suas definições continuam mudando, embora suas fronteiras permaneçam incertas.

%Além disso, não há um verdadeiro sistema de classificação geral reconhecido por todos: essas classificações são subjetivas e compartilhadas principalmente por pequenos grupos de usuários, para um conjunto definido de videogames.
%Várias tentativas de construir um sistema unificado a partir dessas classificações empíricas existem, mas nenhum desses sistemas baseados em acadêmicos ou de designers conseguiu produzir uma classificação reconhecida ainda... A partir dessa falta de classificação de videogame adequada para cada título lançado, levante a necessidade de explorar novas abordagens de classificação.




\section{Engenharia de Sotfware}\label{sec:Engenharia}

%\section{Learning Analytics}\label{sec:LA}

