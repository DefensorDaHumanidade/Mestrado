%!TEX root = ../Principal.tex
\chapter{Introdução}\label{Capitulo:Introducao}

``We evaluate a \textbf{multifaceted policy intervention} attempting to jumpstart adolescent women’s empowerment in Uganda'' ... ``Strikingly, the share of girls reporting sex against their will drops by close to a third and aspired ages at which to marry and start childbearing move forward.'' \cite{bandiera2017women} [BRAC-ELA as a tool to aid womens’ empowerment]

``The interaction between violence and education operates in both directions, which means education can be used as an instrument to reduce the prevalence of violence. In Uganda, for example, a \textbf{programme that provided life skills} and vocational training for girls who had been forced into sexual acts, led to substantially fewer of these girls being victims of sexual abuse – an impact largely attributed to acquired skills''  \cite{owidviolenceagainstrightsforchildren} (Esse artigo referencia o de cima)


Formas de combate a violência sexual (\textbf{PROGRAMAS [AULAS], EXAMES CLINICOS, OBSERVAÇÕES NO COMPORTAMENTO}):

\begin{itemize}
  \item Criança denuncia avô por abuso após aula sobre violência sexual no Paraná. \cite{central2019crianca} [Proerd, avó acareciava ela]
  \item Criança escreve bilhete após palestra em escola de MT e denuncia pai: 'Já fui abusada pelo meu pai, isso pode ser denúncia?' \cite{lidiane2018crianca} [Proerd, pai abusava ela]
  \item Mãe descobre que filha de 5 anos foi estuprada ao levar menina em pediatra de RO \cite{jonatas2018crianca} [Exames de rotina, medica constatou abuso pelo primo de 13 anos]
\end{itemize}

%REVISAR A CITAÇÃO, PELO QUE PARECE, ESSE TIPO DE CITAÇAO VAI COMO NOTA DE RODAPE E NAO NAS REFERENCIAS... Basta dizer: 'Disponível em: <https://oglobo.globo.com/.......'





``Abuso sexual: consiste em todo ato ou jogo sexual, relação heterossexual ou homossexual cujo agressor está em estágio de desenvolvimento psicossexual mais adiantado que a criança ou o adolescente. Tem por intenção estimulá-la sexualmente ou utilizá-la para obter satisfação sexual. Apresenta-se sobre a forma de práticas eróticas e sexuais impostas à criança ou ao adolescente pela violência física, ameaças ou indução de sua vontade. Esse fenômeno violento pode variar desde atos em que não se produz o contato sexual (voyerismo, exibicionismo, produção de fotos), até diferentes tipos de ações que incluem contato sexual sem ou com penetração. Engloba ainda a situação de exploração sexual visando lucros como é o caso da prostituição e da pornografia.'' \cite{saude2002notificacao} [Essa referencia também explica um pouco sobre o conselho Tutelar]


``A resistência da criança ao agressor é pouca, tornando-a uma presa ao sistema relacional patológico, adaptando-se a ele.'' ... Alguns sinais de abuso são: masturbação excessiva, hematomas, brincadeiras sexuais... ``Os profissionais de saúde, em particular os pediatras, não conseguindo perceber esses sinais, subdiagnosticam essa ocorrência por uma série de razões, dentre elas desconhecimento sobre o assunto, falta de sensibilização e capacitação, auxiliando na manutenção do segredo familiar.'' \cite{pavao2013impasse}


Aqui abaixo vemos a \textbf{estratégia da Alemanha} em produzir pornografia infantil falsa:
\begin{itemize}
  \item https://www.zdf.de/nachrichten/heute/lambrecht-will-ermittlern-herstellung-gefakter-kinderpornografie-erlauben-100.html

  \item https://www.dw.com/en/germany-plans-to-use-fake-child-porn-to-snare-pedophiles/a-51361810

  \item https://www.terra.com.br/noticias/alemanha-planeja-usar-pornografia-infantil-falsa-para-capturar-pedofilos,869a166ee7af97bb44f30200b7f93597y5krakph.html
\end{itemize}

``Two widespread forms of sexual assault prevention efforts have been extensively studied and disseminated, namely, \textbf{offender “management” and educational programs} delivered, for the most part, in school settings.''
``The second most frequent approach, primary prevention, involves universal educational programs generally delivered in schools and aimed at potential victims. In the majority of cases, these universal programs also intervene in the individual preventive sphere and more infrequently in the family or societal sphere.''
``27 studies, revealed that programs are effective at building children’s  knowledge about sexual abuse and their preventive skills. The second of those two meta-analyses further demonstrated that programs are more effective if they are longer in duration (four sessions or more), if they repeat important concepts, if they provide children with multiple opportunities to actively practice the taught notions and skills, and if they are based on concrete concepts (what is forbidden) rather than abstract notions (rights or feelings)'' [Ele conclui que há evidência que corroborem a eficácia, mas destaca que não há como afirmar] ``this approach has also been criticized since it places the responsibility of prevention in the hands of children'' [Eu também faço isso]... [por fim, ele diz para não descartar \textbf{campanhas, kit educacionais, capacitação de profissionais, participação dos pais, etc}] \cite{collin2013lessons}



``Assim, as autoras reforçam a importância e a necessidade de os \textbf{professores receberem treinamento especializado} para identificar e intervir nesses casos, já que muitas professoras apresentam apenas um conhecimento superficial sobre o tema, buscam informações em meios não apropriados e não tem clareza sobre os procedimentos que devem tomar'' [Outra estratégia é o treinamento especializado de professores] ``Uso de \textbf{vídeos educativos, oficinas, palestras com profissionais} de diferentes áreas (direito, psicologia, etc) são algumas das alternativas que podem ser utilizadas. Muitas vezes, a educação sexual na escola restringe-se a simples aulas de anatomia e fisiologia dos órgãos sexuais e apresentação de doenças sexualmente transmissíveis.'' ... ``Certamente, muitos alunos seriam beneficiados por uma explicação que iria além da biologia, incluindo relações de poder, sentimentos, saúde e lei.'' ``Um fator abordado por Sanderson (2005) é o de que o abusador, antes de aliciar a vítima, alicia os adultos. Somente conquistando a confiança dos adultos que cuidam da criança é que ele consegue as oportunidades para que o abuso aconteça. Em muitos casos, o processo de conquistar a confiança da família pode durar muito tempo, o que faz com que o abusador obtenha da família uma credibilidade que mais tarde vai dificultar ainda mais a revelação por parte da vítima.'' ``Em se tratando de abuso sexual infantil, o \textbf{TP (treinamento de pais)} pode ser utilizado de forma que conscientize os pais sobre os cuidados necessários para que seus filhos tenham um risco menor de sofrer esse tipo de violência, tanto em casa como na rua.'' \cite{pelisoli2010prevenccao}


``CSA (Child sexual abuse) is associated with emotional and behavioral problems, as well as factors such as substance abuse that increase risk for mental and physical illnesses over the life course (outra referencia aqui)'' ``societal costs of CSA are high. These include expenses associated with \textbf{offenders’ prosecution, incarceration, and monitoring; victims’ medical and psychiatric costs;} effects on victims’ families, relationships, and school and workplace performance; victims’ quality of life; and reduced life expectancy (outra referencia aqui)'' \cite{mendelson2015parent}

\begin{enumerate}
  \item \cite{mendelson2015parent}

  \item .[Justice System Restrictions] = ???????????????????

  \item .[Advocacy and Media Campaigns] = Campanhas governamentais (Darkness to Light, Stop It Now! e Prevention Project Dunkelfeld)

  \item .[Youth-Serving Organizations] => código de conduta????

  \item .[School-Based Programs] = AULAS (PROERD)

  \item .[Treatment of Offenders] = Gestão de Infratores

  \item .[Treatment of Victims] = Tratamento psicológico (centros de tratamento)
  
  \item PROPOSTA DO ARTIGO [Parent-Focused Prevention] = Treinamento de Pais (TP)
\end{enumerate}

``A recent study found close to 90\% of offenders imprisoned for CSA had no history of prior sexual offenses'' \cite{mendelson2015parent}


``Existe una gran variedad de opciones metodologicas al alcance de los usuarios. Dentre de estas, las mas utilizadas han sido los \textbf{materiales impresos, los videos o materiales audiovisuales, las charlas, las representaciones teatrales y el role playing}'' (corrigir erros do espanhol) [Esse artigo é bom, pois fala dos toque bons, toque ruins, partes íntimas, etc] ... ``el abusador impone a el nino ley de silencio (segredo)'' ... ``los programas deberian poner el acento en transmitir a los ninos la importancia de divulgar el abuso y no en pedirles que se nieguen y sean capaces de deternerlo'' [LEMBRA DO JOGO TRIALHA DA PROTEÇÃO, ao completar a criança recebe o título de 'PROTEGIDO'] \cite{martinez2011prevencion} 

TRILHA DA PROTEÇÃO: \cite{meyer2017analise}



David Finkelhor, defende duas estratégias: \textbf{[offender management and school-based educational programs}] ``All states now have electronic sex offender registries. One goal of these registries is to allow more rapid apprehension of re-offenders; another is to prevent crime by deterring existing and future offenders. Some observers, though, argue that registration, like a lot of offender management practices, makes it harder for offenders to reintegrate into society and violates the rights of those who have already paid their debt to society, particularly those forced to register retroactively'' ... ``But though the study linked registration with reduced offending among first-time offenders, it found increased offending among those who were already registered, suggesting a possible boomerang effect from the stigma (increased difficulty finding jobs and housing, for example)'' \cite{finkelhor2009prevention}


------------------  gestão de criminosos (estrategia 1)

\begin{enumerate}
  \item \cite{finkelhor2009prevention}

\item .[Offender Registration] = Dados de criminosos já soltos guardando seus registros (mais fácil de fazer a busca em caso e reincidência)

\item .[Community Notification] = Lei de Megan (informar os vizinhos)

\item .[Mandatory Background Checks] = Entrevistas de trabalho notificadas (impossibilitanto o trabalho com crianças para abusadores)

\item .[Residency Restrictions] = lei de Jessica (proibe os criminosos de acessarem determinados locais, etc)

\item .[Sentence Lengthening and Civil Commitment] = Alongamento de sentenças...

\item .[Enhanced Detection and Arrest] = aumento dos esforços policiais para divulgar, investigar e prender criminosos

\item .[Mental Health Treatment] = terapias e tratamentos para criminosos

\item .[Community Reintegration and Supervision] = Circles of Accountability and Support (CoSA) grupos de voluntários com supervisão profissional para apoiar os agressores sexuais à medida que se reintegram à sociedade após serem libertados do encarceramento.
\end{enumerate}

--------- programas educacionais nas escola (estrategia 2)

``One central goal has been to impart skills to help children identify dangerous situations and prevent abuse'' \cite{finkelhor2009prevention} [formas idesejadas de toques (toques bons e ruins)]


[PROGRAMA 1] \textbf{Talking about Touching} program =  focuses on teaching children basic skills designed to help them keep safe from dangerous or abusive situations. \cite{finkelhor2009prevention} %(https://www.cfchildren.org/wp-content/uploads/resources/previous-programs/talking-about-touching/tatPreKTeachers.pdf) 


[PROGRAMA 2] CAP (\textbf{Child Assault Prevention}) \cite{finkelhor2009prevention}

CRITICAS DOS PROGRAMAS: ``perhaps psychologically harmful to place the responsibility for preventing abuse on the shoulders of children.'' \cite{finkelhor2009prevention}

POSITIVO: ``Do children learn the concepts? Many studies summarized in a variety of reviews find that children of all ages acquire the key concepts being taught.''\cite{finkelhor2009prevention} .. ``An international meta-analysis found that children of all ages who had participated in an education program were six to seven times more likely to demonstrate protective behavior in simulated situations than children who had not.'' ``Analysts have not found that exposure to the program makes children more likely to misinterpret appropriate physical contact and make false allegations.'' [ele diz também que crianças passam a usar as termologias mais corretas para alguns partes do corpo]   ....... [o artigo tambem cita programas de prevenção de drogas, gestação, bullyng e que a literatura reforça que essas estrategias de prevenção funcionam] [ele tambem fala do “burden of responsibility” ] ``\textbf{school-based education programs} have proven to be a successful primary prevention strategy in other domains, some closely related to sexual abuse prevention'' \cite{finkelhor2009prevention}
 
Essa artigo fala que o imperador romano Tibério tinha relações com crianças. E também comenta sobre a primeira monografia na área 'Étude médico-légale sur les sevices et mauvais traitements exercés sur des enfants' de Ambroise Tardieu lembrando que antes disso o médico já tinha outros escritos sobre o assunto. \cite{aded2006abuso}

abusos físico, sexual, psicológico, negligência etc. Delas, não se sabe qual é a mais danosa. \cite{aded2006abuso} 

Os tipos de abuso contra crianças mais comuns e de mais fácil detecção médico-legal são a violência física e a sexual. \cite{aded2006abuso} 

Direito das crianças = 1924, pela Convenção de Genebra sobre os direitos da criança, estendida pela Convenção Internacional das Nações Unidas de 1959 e ratificada em 1990 pelos países signatários \cite{aded2006abuso} 

Por mais que hajam mais denuncias de meninas...... ``. A subnotificação das corrências envolvendo o sexo masculino não pode ser descartada'' \cite{aded2006abuso}.[por isso que o jogo deve ser ministrado a ambos os gêneros]

Na Africa, ``as crianças correm grande risco de contaminação pelo vírus HIV''.. existe a crença que os portadores serão limpados da doença. \cite{aded2006abuso}

``Estudo publicado nos Estados Unidos em 1994, com base no ano de 1993, revelou que 85\% a 90\% dos pacientes com problemas psiquiátricos foram vítimas de algum tipo de mau-trato na infância, com predominância do abuso sexual''\cite{aded2006abuso} [na pagina 207 ele lista alguns sintomas]

``Fatores como a não-notificação das ocorrências às autoridades policiais, por medo de represálias ou do estigma social, dificultam o conhecimento do desfecho desses casos'' [é preciso tirar esse medo das crianças] \cite{aded2006abuso} ...ele deixa claro que no caso de relações com conhecidos, a criança não sabe que os atos praticados são incorretos [é preciso ensinar isso para as crianças]

``Cabe ressaltar que algumas crianças, apesar do sofrimento causado pelos maus-tratos, passam por essa experiência sem apresentar o quadro de seqüelas descrito pela literatura especializada'' \cite{aded2006abuso} 

``A maior parte de abusos sexuais confirmados em crianças impúberes não evidenciam lesões, ou apenas revelam achados inespecíficos''\cite{aded2006abuso} 


``Si bien es difícil establecer criterios generales sobre las consecuencias específicas del abuso sexual infantil, a corto plazo se destacan alteraciones de sueño, trastornos alimenticios, miedo generalizado, rabia y hostilidad, relaciones sexuales precoces, curiosidad sexual excesiva, masturbación compulsiva y dificultades en el rendimiento escolar'' \cite{mariscal2003programa}

O artigo fala da ``síndrome de la mercancía dañada'' [dando alguns sintomas do abuso] [O artigo tambem fala da três prevenções (primaria, segundaria, terciaria)] \cite{mariscal2003programa}

Um dos maiores problemas da``...Bolivia es la carencia de programas de prevención'' \cite{mariscal2003programa}


``este programa de prevención está destinados a niños y niñas preescolares, para actuar antes de que el abuso se presente, favoreciendo la denuncia por parte de las víctimas, ahorrando largos y costosos períodos de tratamiento y considerando factores de riesgo específicos para esta población.'' [Em um momento ele fala de sobre sobre partes íntimas e toques bom e ruins] \cite{mariscal2003programa}

[ESSE TRABALHO PROPOEM \textbf{PROGRAMAS DE PREVENÇÃO AO ABUSO PARA (CRIANÇAS, PAIS, PROFESSORES)}] \cite{mariscal2003programa} [APARENTEMENTE NÃO FORAM IMPLEMENTADOS, PELO MENOS NÃO COM OS NOMES DEFINIDOS NO ARTIGO]


``Estudos  sobre  a  incidência  e  a  prevalência  do  ASI  mostram  que  esse  é  um fenômeno  mundial  endêmico  e  que  demanda  políticas  e  estratégias  de  combate  e enfrentamento.'' \cite{pinto2017avaliaccao}

Habitos, costumes, culturas ``impedem que os pais conversem abertamente sobre a sexualidade infantil'' ...  [É POR ISSO QUE A ESCOLA]``a  escola  deve  ser  considerada  como  um  espaço  privilegiado  para ações  de  combate  a  todo  tipo  de  violência  contra  a  criança,  inclusive  o  ASI,  a  partir  do trabalho  de  conscientização  com  os  pais,  educadores'' \cite{pinto2017avaliaccao}

``... (outra citação) trabalho  em  que  os  pais  são informados  e  orientados  sobre  a  definição,  a  frequência,  as  estratégias  dos  agressores, consequências, entre outras características do ASI, é possível desenvolver determinadas competências  que  lhes  permitam  enfrentar  de  forma  adequada  situações  perigosas  e reduzir  o  índice  de  crianças  abusadas  em  suas  comunidades.'' [IMPORTANTE TOMAR CUIDADO, POIS METADA DOS ABUSOS VEM DE RESPONSAVEIS] \cite{pinto2017avaliaccao}

[Esse artigo fala mais de um \textbf{programa de educação para pais} (ESCLARECER SOBRE A ASI)]

``importante  destacar  que  a  prevenção  na  área  deve  sempre  envolver  um trabalho interdisciplinar e intersetorial, estimulando a parceria entre os vários segmentos e instituições   sociais,   como   Saúde,   Educação,   Justiça'' \cite{pinto2017avaliaccao}


------------------- 
\cite{planalto13431}


[VALOR PROBATÓRIO DA DENUNCIA] - Depoimento especial (ou depoimento sem dano) Art 8


``A Jurisprudência pátria é assente no sentido de que, nos delitos de natureza sexual, por frequentemente não deixarem vestígios, a palavra da vítima tem valor probante diferenciado. (REsp. 1.571.008/PE, Rel. Min. RIBEIRO DANTAS, 5ª Turma, Dje 23/2/2016).'' 
...
``Nos crimes contra a dignidade sexual, uma vez considerada a relevância do depoimento da vítima em harmonia com o contexto fático-probatório dos autos, as pequenas contradições nas suas declarações são insuficientes para invalidá-las,''
...
``Nos crimes contra a dignidade sexual, quase sempre praticados às escondidas, a palavra da vítima ganha especial relevo, mormente, como no caso concreto, quando coerente, sem contradições e em consonância com as demais elementos colhidos nos autos''
...
STJ tem entendido ainda que "a ausência de laudo pericial não afasta a caracterização de estupro, porquanto a palavra da vítima tem validade probante, em particular nessa forma clandestina de delito, por meio do qual não se verificam, com facilidade, testemunhas ou vestígios".

Lei 1.3431 de 2017, ler artigo 21 [PRISÃO PREVENTIVA] - “depoimento sem dano” [a criança ou o adolescente vítima ou testemunha de violência tem direito a pleitear, por meio de seu representante legal, medidas protetivas contra o autor da violência] - [Art 9, sem contato com o acussado] \cite{planalto13431}

-------------------- 


``Os Conselhos Tutelares estão para a violência sexual infantil e adolescente, como as equipes de resgate para os primeiros socorros.'' \cite{caccia2014conselheiros}

``Para nós o tabu apresenta dois significados opostos: o do sagrado e consagrado e o do inquietante, perigoso, proibido e impuro... As restrições tabus são algo muito distintas das proibições puramente morais ou religiosas. Não emanam de nenhum mandamento divino, senão que extraem de si próprias sua autoridade. (FREUD, 1967, p.520).''

``O enfrentamento do violência sexual no âmbito dos \textbf{órgãos públicos estatais e federais ocorre em forma de campanhas de mobilização da cidadania}, através dos meios presentes de comunicação. Nas cidades, essas campanhas chegam através de chamadas, em emissoras de televisão, pela distribuição de panfletos e exposição de mensagens, de propagandas escritas, nas ruas, ou breves alertas nas emissoras de rádio. Também pela divulgação do \textbf{número telefônico 181} , que é reservado para denúncias dessa prática delitiva.'' ... ``enfrentamento da prática de abuso sexual que exige a presença de agentes vinculados ao Sistema Único de Assistência Social-SUAS, ao Sistema Único de Saúde-SUS, ao Sistema Nacional de Educação e às unidades locais de Segurança Pública'' [o artigo tambem falo do SIPIA-CT Web, CREAS e do CRAS] \cite{caccia2014conselheiros}

É importante lembrar que existe diferença entre ``distinção entre ações governamentais voltadas ao enfrentamento da exploração sexual e ações voltadas à prevenção do abuso sexual.''  \cite{caccia2014conselheiros}



[esse artigo tem uns graficos legais, mas antigos.. (SEPARAÇÃO POR RELIGIÃO, ESCOLARIDADE)] [declarações ESPONTANEAS OU NAO DAS CRIANÇAS: enfatizando a importancia de questiona-las] PERGUNTAAAAA: será que o jogo deveria questionar a criança?????????? \cite{cardoso2016abuso} ....tem mais coisas interessantes nesse artigo!!!!!!



\chapter{Resultados de comandos}\label{cap_exemplos}

``Com base nas dissertações e nos artigos selecionados, foi possível verificar que há escassez de pesquisas e publicações sobre as práticas de jogos eletrônicos na infância, em especial na primeira.''  \cite{cotonhoto2016pratica}
